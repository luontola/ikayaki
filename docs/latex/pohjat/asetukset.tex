%
% T�ss� esimerkkipohjassa oletetaan, ett� projektiryhm�n nimi on esim.
% Muuta tiedostoa vastaamaan projektisi tietoja.
%

% Times-pakkaus on hy�dyllinen, koska sen avulla tulee siist� PDF:��.
\usepackage{times}

% Seuraavaan pit�� muuttaa projektiryhm�n nimi.
\defprojectgroup{SQUID}

% Ryhm�n j�senet. Viimeisen j�senen j�lkeen ei tule \\-merkint��
\defgroupmembers{
  \membername{Mikko Jormalainen}\\
  \membername{Samuli Kaipiainen}\\
  \membername{Aki Korpua}\\
  \membername{Esko Luontola}\\
  \membername{Aki Sysm�l�inen}
}

% Projektin asiakkaan nimi. Jos asiakkaita on useita, nimet��n heist� yksi
% ensisijaiseksi asiakaskontaktiksi.
\defprojectclient{Lauri J. Pesonen\\
  \membername{Fabio Donadini}\\
  \membername{Tomas Kohout}
}

% Projektiryhm�n kotisivu.
\defprojecthomepage{http://www.cs.helsinki.fi/group/squid/}

% Projektin johtoryhm� eli vastuuhenkil�t.
\defprojectmasters{
  \mastername{Juha Taina}\\
  \mastername{Jenni Valorinta}
}

%
% Omat, kaikille projektin dokumenteille yhteiset asetukset voi laittaa
% t�nne.
%
