% est�mme ihme "underfull \hbox (badness 10000)" -varoitukset (ei hajua mist� tulevat)
\hbadness=10000

% makrot dokumentoinnin generoimiseksi

% Macros for generating design documents and .java files

% MACROS FOR CLASS DOCUMENTATION
\newcommand{\getClass}[0]{}
\newcommand{\beginClass}[1] {
	\subsubsection{#1}
	
	\renewcommand{\getClass}[0]{#1}
	\label{class:#1}
	
	\begin{hyphenrules}{nohyphenation}
	\begin{tabular}{p{3cm}p{11cm}}
}
\newcommand{\classComment}[1] {
	\multicolumn{2}{p{14cm}} {
		#1
	}\\
}
\newcommand{\classPackage}[1] {
	\textbf{Package} & #1 \\
}
\newcommand{\classDeclaration}[1] {
	\textbf{Declaration} & #1 \\
}
\newcommand{\classExtends}[1] {
	\textbf{Extends} & #1 \\
}
\newcommand{\classImplements}[1] {
	\textbf{Implements} & #1 \\
}
\newcommand{\classCreatedBy}[1] {
	\textbf{Created by} & #1 (\ref{class:#1}) \\
}
\newcommand{\classUses}[1] {
	\stepcounter{classUsesCounter}
	\textbf{Uses \arabic{classUsesCounter}} & #1 (\ref{class:#1}) \\
}
\newcommand{\classSubclass}[1] {
	\stepcounter{classSubclassCounter}
	\textbf{Subclass \arabic{classSubclassCounter}} & #1 (\ref{class:#1}) \\
}
\newcommand{\classPatterns}[1] {
	\textbf{Design patterns} & #1 \\
}
\newcommand{\classEvent}[2] {
	\stepcounter{classEventCounter}
	\textbf{Event \Alph{classEventCounter}} & \textit{#1} - #2 \\
}
\newcommand{\closeClass}[0] {
	\end{tabular}\\
	\end{hyphenrules}
	
	\setcounter{classUsesCounter}{0}
	\setcounter{classSubclassCounter}{0}
	\setcounter{classEventCounter}{0}
	
	\renewcommand{\isFirstField}[0]{ \large\textbf{Fields of \getClass}\normalsize }
	\renewcommand{\isFirstMethod}[0]{ \large\textbf{Methods of \getClass}\normalsize }
}
\newcounter{classUsesCounter}
\newcounter{classSubclassCounter}
\newcounter{classEventCounter}
\newcommand{\isFirstField}[0]{}
\newcommand{\isFirstMethod}[0]{}


% MACROS FOR CLASS TABLE OF CONTENTS
\newcommand{\beginToc}[0] {
	\begin{hyphenrules}{nohyphenation}
	%\begin{tabular*}{14cm}{lr}
	\begin{tabular*}{14cm}{@{\extracolsep{\fill}}lr}
	\textbf{Method} & \textbf{Page} \\
}
\newcommand{\tocMethod}[1] {
	#1 & \pageref{method:\getClass.#1} \\
}
\newcommand{\closeToc}[0] {
	\end{tabular*}\\
	\end{hyphenrules}
}


% MACROS FOR FIELD DOCUMENTATION
\newcommand{\getField}[0]{}
\newcommand{\beginField}[1] {
	\isFirstField
	\renewcommand{\isFirstField}[0]{}
	
	\renewcommand{\getField}[0]{#1}
	\label{field:\getClass.#1}
	
	\begin{hyphenrules}{nohyphenation}
	\begin{tabular}{p{3cm}p{11cm}}
}
\newcommand{\fieldDeclaration}[1] {
	\multicolumn{2}{p{14cm}} {
		\hspace{-0.32cm}
%		\large{\textsf{#1}}
		\textsf{#1}
	}\\
}
\newcommand{\fieldValue}[1] {
	\textbf{Default value} & #1 \\
}
\newcommand{\fieldComment}[1] {
	\multicolumn{2}{p{14cm}} {
		#1
	}\\
}
\newcommand{\closeField}[0] {
	\end{tabular}\\
	\end{hyphenrules}
}


% MACROS FOR METHOD DOCUMENTATION
\newcommand{\getMethod}[0]{}
\newcommand{\beginMethod}[1] {
	\isFirstMethod
	\renewcommand{\isFirstMethod}[0]{}
	
	\renewcommand{\getMethod}[0]{#1}
	\label{method:\getClass.#1}
	
	\begin{hyphenrules}{nohyphenation}
	\begin{tabular}{p{3cm}p{11cm}}
}
\newcommand{\methodDeclaration}[1] {
	\multicolumn{2}{p{14cm}} {
		\hspace{-0.32cm}
%		\large{\textsf{#1}}
		\textsf{#1}
	}\\
}
\newcommand{\methodComment}[1] {
	\multicolumn{2}{p{14cm}} {
		#1
	}\\
}
\newcommand{\methodParam}[2] {
	\stepcounter{methodParamCounter}
	\textbf{Parameter \arabic{methodParamCounter}} & \textit{#1} - #2 \\
}
\newcommand{\methodReturn}[1] {
	\textbf{Returns} & #1 \\
}
\newcommand{\methodThrows}[2] {
	\textbf{Throws} & \textit{#1} - #2 \\
}
\newcommand{\closeMethod}[0] {
	\end{tabular}\\
	\end{hyphenrules}
	\setcounter{methodParamCounter}{0}
}
\newcounter{methodParamCounter}




\section{Introduction}
\label{sec:intro}

This document describes how this software (Ikayaki) is planned to be tested properly. Mainly this document concentrates on describing methods used for testing and test cases. It is important that all members of team to make tests in same way. This lowers possibility of testing conflicts and helps on integration test phase.


\section{Overview of testing the system}
\label{sec:overview}

Because program will be used to control a magnetometer, testing will be more important than in normal software engineering student projects. We will do unit testing for each class, integrate testing to program and use separate squid emulator to test squid interface system. 

In unit testing each class is tested independently. Unit testing will be done by using JUnit. Every programmer will test his own classes. Class should be tested when it is finished and corrected before integration test begins. 

Integration testing tests interfaces between classes. It will be done by going through all user interface protos and checking that all sections in requirements document can be done. Some critical sequences which are done many times with program should be done too.

Squid interface integration testing is done simulating real system with emulator. It will be done using Squid-emulator 
before testing it with real magnetometer. Squid-emulator runs in different machine 
and is connected by few (2-3) Serial I/O cables. Squid-emulator will be tested with old program (2G) same way before testing 
Ikayaki-system so that it will have all same tested properties which old program have and both systems have same results with squid 
emulator.

To verify that old program and new program works same way, we will do critical measurement with old program and emulator, save emulators 
log file and then use emulator with that log file and do same critical measurement with new program and see that both have same results.

If Rita testing utility is easy enough to use it will be used in testing. Tests will be constructed in such way that every line of code is visited at least once.


\section{Testing}
\label{sec:test}

\subsection{Unit Testing}
\label{sec:unit}

Unit Testing is done for each class separately. Classes tested with JUnit have it's own ClassNameTest.java class in test-directory and gui-components are tested manually. They should be done before and during coding class. All test cases must be executable after classes are ready. Every class should be tested succesfully before integration tests.

All test cases are listed in sections \ref{sec:junit} and \ref{sec:guicase}.

\subsection{Integration Testing}
\label{sec:integration}

Integration testing is mostly done after unit tests are passed for all classes. Testing is done
using squid emulator at first and finally with real Squid system. Graphical User Interface is
mainly tested in Unit testing and is only looked that all components work together as Squid
Interface is tested.

Integration testing for GUI-components is done using top-down method.
Testing is done partly during implementation phase, when new components
are added to program. After adding component to program it is tested to
work properly with other components.

Graphical User Interface is tested using all use cases from \ref{sec:intcase}. If one fails, it is
corrected immediatelly and all use cases must be done again. When all use cases are done
without errors this phase is ready.

Squid Interface testing is done with Emulator. Use cases are Automatic Measurement, Thellier
Measurement and Manual Measurement with all variations (see \ref{sec:intcase}). In first
phase we use old program with emulator, save its log file and take results. After that we
run it with new program and compare results. If they are not same, corrections
are made immediatelly to new program and test is run again. This is done until results are
same for all use cases.

\subsection{Squid Emulator}
\label{sec:emulator}

Squid emulator is tested with old program. Use cases are Automatic Measurement, Thellier Measurement and Manual Measurement with all variations (see Requirements Document). When all use cases can be done with emulator it is ready enough for integration testing.

This should be done so that Old Program has same results with SQUID-system and emulator for all use cases. But we don't have resources for this. So this test only tells us that Old and New program works same way with squid emulator.


\section{JUnit test cases}
\label{sec:junit}

Here are listed JUnit test cases for classes.

JUnit is simple java-based framework for testing your java classes. We use it in this project for Unit Testing. For more information visit http://junit.sourceforge.net/.

First you need to download JUnit from http://sourceforge.net/project/showfiles.php?group \_id=15278 and extract it to directory (different than java directory) and set classpath for it.

Then you must write test class for every class to be tested. Test classes extend TestCase. They will have test methods, one for each Test Case. Test class also need suite()-method and Main for run. Sample test class:
\begin{verbatim}
  import java.util.*;
  import junit.framework.*;

  public class SimpleTest extends TestCase {

  //Simple Test Case

  public void testEmptySimple() {
        Simple simple = new Simple();
        //New Simple must be empty
        assertTrue(simple.isEmpty());

  }

  public static Test suite() {
        return new TestSuite(SimpleTest.class);
  }

  public static void main(String args[]) {
        junit.textui.TestRunner.run(suite());
  }
}
\end{verbatim}

Type "java junit.swingui.TestRunner SimpleTest" to run test cases for Simple.


\beginTestedClass{SampleClass}

\beginCase{Test1}
\condition{Cond1}
\condition{Cond2}
\closeCase

\beginCase{Test2}
\condition{Cond1}
\closeCase

\closeTestedClass


\section{GUI-component test cases}
\label{sec:guicase}

Here are listed test cases for gui-classes, these are done manually just clicking and changing values.

\beginTestedClass{Ikayaki/MainViewPanel}

\beginCase{Starting the program}

These have been tested with \texttt{test-ikayaki.bat}.

\condition{Starts up correctly, everything works as supposed}
\testDescription{No parameters}\testResultTrue
\testDescription{Parameter "*/?" (which doesn't exist)}\testResultTrue
\testDescription{Parameter "projects/test.ika" (which exists, should open it)}\testResultTrue
\testDescription{No parameters and without config file}\testResultTrue
\testDescription{No parameters, with config file left from previous}\testResultTrue
\testDescription{Parameter "projects/test.ika" (which exists, should open it), and without config file}\testResultTrue

\closeCase

\closeTestedClass


\beginTestedClass{ProjectExplorerPanel}

\beginCase{Directory text field}
\condition{Shows autocomplete popup when typing and typed text matches to one or more directory}\testResultTrue
\condition{Shows directory history on down-arrow click}\testResultTrue
\condition{Too long lines in popup are shortened}\testResultTrue
\condition{Tries to change directory when on enter press or mouse-click to popupmenu (UC21)}\testResultTrue
\condition{Only allows correct and existing directories (flashes red on change-attempt if invalid)}\testResultTrue
\closeCase

\beginCase{Browse-button}
\condition{Opens a dir chooser dialog}\testResultTrue
\condition{Changes to chosen directory if "Open" selected; updates text field (UC21)}\testResultTrue
\closeCase

\beginCase{Project file table}
\condition{Shows project files in current directory}\testResultTrue
\condition{Only shows project files (.ika and Calibration/AF/Thellier/Thermal)}\testResultTrue
\condition{Hilights open project}\testResultTrue
\condition{Hilights (with different color) currently measuring project}\testResultTrue
\condition{Loads project on project file mouseclick (UC20)}\testResultTrue
\condition{Show export menu for clicked file on right-click}\testResultTrue
\condition{Export to selected file if any chosen (UC13-UC15)}\testResultTrue
\condition{Sorts data on header click}\testResultTrue
\condition{Updates data if changed (modifield-column, that is)}\testResultTrue
\closeCase

\beginCase{"Create New" -components}
\condition{Only allows correct, non-existing files; flashes red if invalid}
\testDescription{/ or $\backslash$}\testResultFalse
\condition{If valid filename, creates it, updates table with created file selected (UC19)}\testResultTrue
\closeCase

\closeTestedClass

\beginTestedClass{CalibrationPanel}

\beginCase{Calibration project table}
\condition{Shows (and only shows) calibration project files in current directory}
\testResultTrue
\condition{Hilights open project}
\testResultTrue
\condition{Hilights (with different color) currently measuring project}
\testResultTrue
\condition{Loads project on project file mouseclick (UC20)}
\testResultTrue
\condition{Doesn't show export menu for clicked file on right-click}
\testResultFalse 
\condition{Sorts data on header click}
\testResultTrue would be better if always in same order
\condition{Updates data if changed (measured- and elapsed-columns, that is)}
\testResultTrue
\closeCase

\beginCase{"Calibrate" -button}
\condition{Does whatever "Single Step" -button does (see MeasurementControlsPanel) (UC8-UC10)}\testResultTrue
\closeCase

\closeTestedClass

\beginTestedClass{MeasurementControlsPanel}

\beginCase{Measuring buttons}
\condition{All of them are enabled/disabled by good manners}\testResultTrue
\condition{On click, all of them execute their measuring command without unjustified complaints (UC1-UC7)}\testResultTrue
\condition{Flash red on click if action can't be done}\testResultTrue
\closeCase

\beginCase{+Z/-Z info/radiobuttons}
\condition{Shows correct orientation for current project}\testResultTrue
\condition{Changes orientation on +z or -z radiobutton click}\testResultTrue
\closeCase

\beginCase{Magnetometer status picture}
\condition{Always shows handler positioin and rotation as supposed...}\testResultTrue
\condition{Updates smoothly \symbol{"5E}\symbol{"5E}}\testResultTrue
\closeCase

\beginCase{Handler manual controls (radiobuttons)}
\condition{Disabled if "normal" measuring in action, enabled otherwise}\testResultTrue
\condition{Executes desired action (if possible; can't tell) (UC11)}\testResultTrue
\condition{Stack move-radiobuttons correctly (never on top of each other or missing)}
\testDescription{Any two handler positions are set (in Settings) to same location (number)...}\testResultFalse
\closeCase

\beginCase{Magnetometer/demagnetizer manual controls (buttons)}
\condition{Disabled if "normal" measuring in action, enabled otherwise if applicable}\testResultTrue
\condition{Enable/disable and update text according to current handler position and rotation}\testResultTrue
\condition{Executes desired action (if possible; flash red if not) (UC11)}\testResultTrue
\closeCase

\closeTestedClass


\beginTestedClass{SettingsPanel}

\beginCase{Magnetometer, Handler and Degausser COM-port combobox}
\condition{Shows all COM-ports on system in alphabetical order}
\testResultTrue{}
\condition{Loads correct COM port from Settings}
\testResultTrue{}
\condition{Doesn't allow same value as in Handler and Magnetometer/Degausser COM-port comboboxes}
\testResultFalse{}
\closeCase{}

%\testResultFalse{}

\beginCase{Save button}
\condition{Only available if changes are made and values are permissible}
\testResultTrue{}
\condition{Unavailable on start}
\testResultTrue{}
\condition{On click closes window and saved data correctly}
\testResultTrue{}
\closeCase{}

\beginCase{Cancel button}
\condition{On click closes window and doesnt save changes}
\testResultTrue{}
\condition{Always available}
\testResultTrue{}
\closeCase{}

\beginCase{Calibration constants}
\condition{Loads data correctly from Settings}
\testResultTrue{}
\condition{Accepts positive and negative decimal numbers}
\testDescription{-0.430002}
\testResultTrue{}
\closeCase{}

\beginCase{Degausser ramp}
\condition{Loads data correctly from Settings}
\testResultTrue{}
\condition{Has only elements: 3,5,7,9}
\testResultTrue{}
\closeCase{}

\beginCase{Degausser delay}
\condition{Loads data correctly from Settings}
\testResultTrue{}
\condition{Has only elements: 1..9}
\testResultTrue{}
\closeCase{}

\beginCase{Acceleration and deceleration fields}
\condition{Loads data correctly from Settings}
\testResultTrue{}
\condition{Accepts only values in range of Integers 0..127}
\testDescription{1000}
\testResultTrue{}
\testDescription{-10}
\testResultTrue{}
\closeCase{}

\beginCase{Velocity and velocity in measurement fields}
\condition{Loads data correctly from Settings}
\testResultTrue{}
\condition{Accepts only values in range of Integers 50..20000}
\testDescription{0}
\testResultTrue{}
\testDescription{2001}
\testResultTrue{}
\closeCase{}

\beginCase{Translation positions fields}
\condition{Loads data correctly from Settings}
\testResultTrue{}
\condition{Accepts only values in range of Integers 0..16777215}
\testDescription{-1}
\testResultTrue{}
\closeCase{}

\beginCase{Right limit}
\condition{Loads data correctly from Settings}
\testResultTrue{}
\condition{Has only elements: plus, minus}
\testResultTrue{}
\closeCase{}

\beginCase{Rotation field}
\condition{Loads data correctly from Settings}
\testResultTrue{}
\condition{Accepts positive and negative Integer numbers}
\testDescription{-1}
\testResultTrue{}
\testDescription{0.1}
\testResultTrue{}
\closeCase{}

\closeTestedClass


\beginTestedClass{SampleClass}

\beginCase{Use case1}
\condition{Cond1}
\testDescription{Description1}
\testResultTrue
\condition{Cond2}
\testResultFalse
\closeCase{}

\closeTestedClass{}


\section{Integration test cases}
\label{sec:intcase}

Here is test case which is done in integration phase to all components at same time. This is done in following order and must be repeated until done succesfully. Variations can be made to ensure that all works in different situations.

% \subsection{GUI test case}
% \begin{enumerate}

\item Open software

Conditions:
\begin{itemize}
\item Last project is opened, if first open then no project.
\item All components are in right place
\end{itemize}

\end{enumerate}

\begin{enumerate}

\item Open software

Conditions:
\begin{itemize}
\item Last project is opened, if available.
\item All components are in right place
\end{itemize}

\item Select Sample holder calibration

Conditions:
\begin{itemize}
\item opens Sample holder calibration project in Measurement Sequence and Information
\item disables Measurement Controls
\item enables calibrate button
\item disables editing sequence
\end{itemize}

\item Click Calibration button

Conditions:
\begin{itemize}
\item Adds new line in Sequence at bottom
\item Starts calibration, progress shown on measurement controls
\item Data is updated correctly in Measurement Details
\item Calibration button changed to stop button
\end{itemize}

\item Calibration finished

Conditions:
\begin{itemize}
\item New data added to Measurement Sequence
\item Stop button is changed to calibration button
\item Last modified is updated Sample Holder
\end{itemize}

\item Select standard sample

Conditions:
\begin{itemize}
\item opens Standard Sample calibration project in Measurement Sequence
\end{itemize}

\item Click Calibration button

Conditions:
\begin{itemize}
\item Adds new line in Sequence at bottom
\item Starts calibration, progress shown on measurement controls
\item Data is updated correctly in Measurement Details
\item Calibration button changed to stop button
\end{itemize}

\item Calibration finished

Conditions:
\begin{itemize}
\item New data added to Measurement Sequence
\item stop button is changed to calibration button
\item Last modified is updated in Standard Sample
\end{itemize}

\item Create new AF project

Conditions:
\begin{itemize}
\item Calibration project is closed
\item new AF project is opened with given name (ProjectInfo,MeasurementSequence)
\item Measurement Controls has enabled buttons Measure,Single Step and disable Stop Now!
\end{itemize}

\item Add project Information

Conditions:
\begin{itemize}
\item All data is accepted
\end{itemize}

\item Load Set

Conditions:
\begin{itemize}
\item Loaded set is correctly added to Sequence
\end{itemize}

\item Click Measure

Conditions:
\begin{itemize}
\item Measure button changed to Pause, disables Single Step, enables Stop Now!
\item Rows are highlighted in Explorer and Sequence
\item Starts measuring first phase correctly (no demag now)
\item Picture of demagnetizer is up-to-date
\item Plots are drawn when one phase is over
\item Continues on next phase and does demag first
\item updates Details and Sequence
\item any sequence cannot be changed 
\end{itemize}

\item Pause button

Conditions:
\begin{itemize}
\item Finish current measure and stops after that
\item Pause button changed to Measure button
\item When finished disables Stop Now!
\end{itemize}

\item Sequence edit

Conditions:
\begin{itemize}
\item Steps not yet measured, can be changed
\end{itemize}

\item Single step button

Conditions:
\begin{itemize}
\item Performs Next phase on measure
\item Disables Measure button and Single Step button, enables Stop Now!
\end{itemize}

\item Measure button

Conditions:
\begin{itemize}
\item Measure button changed to Pause, disables Single Step, enables Stop Now!
\item Rows are highlighted in Explorer and Sequence
\item Continues from Next phase
\item Picture of demagnetizer is up-to-date
\item Plots are drawn when one phase is over
\item Continues on next phase
\item updates Details and Sequence
\end{itemize}

\item Stop Now! button

Conditions:
\begin{itemize}
\item Stops measurement immediatelly.
\item Disables buttons and enables Manual Control
\item What now? Is project Manual or is this only temporary?
\end{itemize}

\end{enumerate}

\beginTestedClass{Thellier Measument}

\beginCase{Open software}
\condition{Last project is opened, if available.}
\condition{All components are in right place}
\closeCase

\beginCase{Open existing Thellier project}
\condition{Project is opened correctly in Information and Sequence}
\condition{Measure button is hidden, Single step enabled and Stop now Disabled}
\closeCase

\beginCase{Add new line in sequence}
\condition{It's inserted correctly.}
\closeCase

\beginCase{Click Single Step button}
\condition{Performs Next phase on Sequence}
\condition{Disables Single Step button, enables Stop Now!}
\condition{Rows are highlighted in Explorer and Sequence}
\condition{Picture of demagnetizer is up-to-date}
\condition{Plots are drawn}
\condition{updates Details and Sequence}
\closeCase

\beginCase{Select next Project}
\condition{Opens project correctly}
\condition{Adds same phase to sequence (as in last project added)}
\closeCase

\beginCase{Click Single Step button}
\condition{Performs Next phase on Sequence}
\condition{Disables Step button, enables Stop Now!}
\condition{Rows are highlighted in Explorer and Sequence}
\condition{Picture of demagnetizer is up-to-date}
\condition{Plots are drawn}
\condition{updates Details and Sequence}
\closeCase

\closeTestedClass

\begin{enumerate}

\item Open software

Conditions:
\begin{itemize}
\item Last project is opened, if available.
\item All components are in right place
\end{itemize}


\end{enumerate}
