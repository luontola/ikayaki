% est�mme ihme "underfull \hbox (badness 10000)" -varoitukset (ei hajua mist� tulevat)
\hbadness=10000

\section{Introduction}
\label{sec:intro}

This document describes how this software (Ikayaki) is planned to test properly. Mainly this concentrates on to describe methods used for testing and to introduce them. And this is important to clear out to all members of team to make testing in same way. This lowers possibilities of testing conflicts and helps on integration test phase.

\section{Overview of testing the system}
\label{sec:overview}

Because program will be used to control a magnetometer, testing will be more important than in normal software engineering student projects. We will do unit testing for each class, integrate testing to program and use separate squid emulator to test squid interface system. 

In unit testing each class is tested independently. Unit testing will be done by using JUnit. Every programmer will test his own classes. Class should be tested when it is finished and corrected before integration test begins. 

Integration testing tests interfaces between classes. It will be done by going through all user interface protos and checking that all sections in requirements document can be done. Some critical sequences which are done many times with program should be done too.

Squid interface integration testing is done simulating real system with emulator. It will be done using Squid-emulator 
before testing it with real magnetometer. Squid-emulator runs in different machine 
and is connected by few (2-3) Serial I/O cables. Squid-emulator will be tested with old program (2G) same way before testing 
Ikayaki-system so that it will have all same tested properties which old program have and both systems have same results with squid 
emulator.

To verify that old program and new program works same way, we will do critical measurement with old program and emulator, save emulators 
log file and then use emulator with that log file and do same critical measurement with new program and see that both have same results.

If Rita testing utility is easy enough to use it will be used in testing. Tests will be constructed in such way that every line of code is visited at least once.

\section{Testing}
\label{sec:test}

\subsection{Unit Testing}
\label{unit}

This is done for each class. Every class has it's own ClassNameTest.java class in test-directory. They should be done before and during coding class. They have to be runnable after class is ready.

\subsection{Integrity Testing}
\label{integrity}

Integrity testing is done after unit test is passed for all classes. Testing is done using squid emulator and trying that use cases can be done with software.

\subsection{Squid Emulator}
\label{emulator}

Squid emulator is tested with old program.