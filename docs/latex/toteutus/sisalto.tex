% est�mme ihme "underfull \hbox (badness 10000)" -varoitukset (ei hajua mist� tulevat)
\hbadness=10000

% nopsa pikkuluokkakaavioiden lis�ysmakro
% pois figuren sis�lt� niin kuvat tulee minne pit��kin, feikataan kuvanumerointi ja kuvateksti
\newcommand{\insertdia}[1] {
	% \begin{figure}[h!]
	\begin{center}
	\includegraphics[scale=0.33]{dia/#1.eps}
	% \caption{#1}
	\refstepcounter{figure}
	\label{fig:#1}
	\nopagebreak \\ \medskip Figure \arabic{figure}: #1 \\
	\addcontentsline{lof}{figure}{#1}
	\end{center}
	% \end{figure}
}

\section{Introduction}
\label{sec:intro}

Blablabla balalaalllblaaaaaaa aaaaarrggrgghh.


\section{Overview of changes from Design document}
\label{sec:overview}

The big picture.


\section{Specific changes from Design document}
\label{sec:changes}

Here we describe how the implementation/production differs from that planned in Design document.


\subsection{Data classes and methods}

Class diagram of data classes is in Figure~\ref{fig:core}.

\subsubsection{Project data}

\insertdia{project_graph}.

Responsible for holding all the measurement data and controlling the SQUID. Most of the GUI classes use the Project class. When the state of the project changes, the Project class fires ProjectEvents and MeasurementEvents to the GUI classes, which in turn will call the Project class to get the changed information.

\subsubsection{Squid interface}

Squid Interface offers the Project class an interface to safely control the SQUID magnetometer. The Squid class holds three classes that handle communication to to three separate parts of the SQUID (Handler, Degausser and Magnetometer).

Classes are Squid, Handler, Magnetometer, Degausser.

\subsubsection{Squid emulator}

Squid Emulator is separate from the rest of the program and it is used only for testing that the Squid Interface works correctly. Biggest change was that this wasnt developed much and its not working as planned. Mainly was used to test that commands are sended correctly and Squid Interface gets answers.

\subsubsection{Serial communication}

SerialIO and classes related to it takes care of the harware layer of serial communication. Using these classes the program communicates with the Degausser, Samplehandler and Magnetometer. SerialIO represents one serial port and when it's created it reserves the port to itself. SerialProperties class includes all the configuration data for the serial port.

\subsubsection{Global settings}

Global properties that are used all around the program. The Settings class provides a global point for retrieving and modifying the properties.

\subsubsection{Utilities}

Utility classes that are used in the program, but do not fit any of the other packages. At the moment includes only the LastExecutor class for thread management, but more classes can be added as necessary.


\subsection{GUI classes and methods}

Class diagram of gui classes is in Figure~\ref{fig:gui}. This section is divided into sections by gui components, each of which has one or more classes.

\subsubsection{Generic GUI components}

ProjectComponent, a generic gui base class which handles registering Project- and MeasurementListeners to new projects, and which every project-dependant gui component subclasses. See Figure~\ref{fig:gui}.

\subsubsection{Main window}

\insertdia{main_view_panel_graph}

Creates main view for GUI. Menubar at top, Statusbar at bottom and makes panels and splitpanels for other GUI components to middle. This also tells other GUI components if someone changes project file.

\subsubsection{Configuration window}

Separate window which is opened from menubar and it updates settings for Squid interface. Used usually only when system is installed to setup it.

\subsubsection{Project Explorer}

Nothing much here; some internal private field renames. Oh and a lot more stuff than planning suggests :)

ProjectExplorerTable is now its own class (as Calibration uses it too), and ProjectExplorerPopupMenu is its inner class. Also has inner class ProjectExplorerTableModel, which, unlike normally (as in the standard way), is quite empty and most of the stuff is in ProjectExplorerTable. Uses StyledWrapper for table row background colors (and Calibration boldface-reminder).

NewProjectPanel is inner class of ProjectExplorerPanel.

\subsubsection{Calibration}

Uses ProjectExplorerTable, which makes Calibration.java really short.

\subsubsection{Project information}

Located at lower left corner of main window.

Contains and allows editing of basic information of sample. Includes
such fields as volume, strike and dip, which are used to make
calculations. Includes also fields whose information is only for users
benefit.

\subsubsection{Sequence and measurement data}

Located at center of main window.

Contains and allows editing of measurement sequence. Whenever
measurement step is finished its data is added here. This data is also
recalculated whenever some field affecting it in project infromation is
changed.

\subsubsection{Measurement details}

Located at middle bottom of main window.

Contains details of ongoing measurement or of row selected in measrurement
sequence. If details of ongoing measurement is shown, they are updated
whenever new measurement data is received and when measurement step is
finished next steps details are shown.

\subsubsection{Measurement controls}

Sadly, ManualControlsPanel had to go into MagnetometerStatusPanel as an inner class, since they share some data and move-radiobuttons.

MagnetometerStatusPanel's updateStatus takes no parameters, as it asks position and rotation directly from Handler. Also, status image is updated at all times once in 50 ms (positions asked with getEstimatedPosition and getEstimatedRotation in Handler).

Here too, a lot more stuff than what planning imply.

ManualControlsPanel's manual control components are disabled whenever there's \textit{any} Squid action, to keep those fragile equipments from messing up.

\subsubsection{Graphs}

Located at lower right corner of main window.

Graph panels visualize the measurement data. MeasurementGraphsPanel listens to MeasurementEvents to update the measurement data in plots. AbstractPlot is an abstract class which implements all the general features of graph plots. IntensityPlot and StereoPlot extend the functionality of AbstractPlot and implement their special drawing features accordingly.


\section{TODO-list from source code}
\label{sec:todo}

\subsection{Ikayaki.java}
        //add(main.getStatusBar(), "South");    // TODO: there is no status bar

\subsection{MeasurementEvent.java}

\subsection{MeasurementListener.java}

\subsection{MeasurementResult.java}

\subsection{MeasurementSequence.java}

\subsection{MeasurementStep.java}

\subsection{MeasurementValue.java}
                        // TODO: should there be the full vectors, or normalized vectors (length=1)?

\subsection{Project.java}
    private static final boolean DEBUG = false;      // TODO: used for testing the measurements without a Squid
        // TODO: use a DTD for the document
        // create document's root element
        return false; // TODO
            out.print(pad(s, Math.max(getName().length(), 8), 1));      // TODO: how should this be calculated?
            // strike
            return false;   // TODO: maybe its better to throw an exception, so that it would be easier to find bugs
            return false;    // TODO: maybe its better to throw an exception, so that it would be easier to find bugs
            return false; // TODO: maybe its better to throw an exception, so that it would be easier to find bugs
            return false;   // TODO: maybe its better to throw an exception, so that it would be easier to find bugs
    // TODO: is this comment even close?
                // TODO: these should be in one Magnetometer method
                    // TODO: these should be in one Magnetometer method

\subsection{ProjectEvent.java}

\subsection{ProjectListener.java}

\subsection{Settings.java}

\subsection{gui/AbstractPlot.java}

\subsection{gui/CalibrationPanel.java}

\subsection{gui/ComponentFlasher.java}

\subsection{gui/DeviceSettingsPanel.java}
            //TODO: check COM ports
            // TODO: doesn't work

\subsection{gui/FittedComboBoxRenderer.java}

\subsection{gui/GenericFileFilter.java}

\subsection{gui/IntensityPlot.java}
        // TODO draw ticks and numbers for y-axis
        // x-axis ticks
        // TODO draw ticks and numbers for x-axis
        // y-axis unit
        // x-axis unit

\subsection{gui/MagnetometerStatusPanel.java}
    // TODO: some way to get the actual max-position?
    // handler positions, read from Settings
    // WARNING: all of these must differ or we have trouble...
        // TODO: WARNING: if two positions are the same, previous one gets replaced
        // TODO: only need to call updateButtonPositions here, but it won't work so now it's called
        // every time in paintComponent
        // TODO: is this needed?
        // TODO: what would be the right place for this call?
        // let Swing erase the background
                // TODO: this wouldn't actually be needed if we get MeasurementEvent every time
                // handler stops, but let's just make sure :)
                // TODO: different speed in measurement zone
            // TODO: as this is the biggest button is measurePanel, changing its text changes whole panel width,
            // and, in the end, whole MagnetometerStatusPanel width; should prevent that from happening

\subsection{gui/MainMenuBar.java}
            //exportProjectMenu.add(exportProjectToSRM);    // TODO: support for SRM files is missing

\subsection{gui/MainStatusBar.java}
        return; // TODO
        return; // TODO
        return; // TODO

\subsection{gui/MainViewPanel.java}
                    // TODO: what should be done now? give error message?
                    //e.printStackTrace();
        // TODO: make this as an Action?

\subsection{gui/MeasurementControlsPanel.java}

\subsection{gui/MeasurementDetailsPanel.java}

\subsection{gui/MeasurementGraphsPanel.java}
        return; // TODO

\subsection{gui/MeasurementSequencePanel.java}
        // TODO: Put these same actions to the program's main menu. Each action might then need to find out the selected rows itself and monitor the ListSelectionModel.

\subsection{gui/MeasurementSequenceTableModel.java}

\subsection{gui/NullableDecimalFormat.java}

\subsection{gui/Plot.java}

\subsection{gui/PositiveDecimalFormat.java}

\subsection{gui/PrintPanel.java}

\subsection{gui/ProgramSettingsPanel.java}

\subsection{gui/ProjectComponent.java}

\subsection{gui/ProjectExplorerPanel.java}
                // TODO: cycling through popup menu list with up/down keys changes directory;
                // it shouldn't, but can't recognize those changed-events from mouse clicks
                // item is File if selected from list, String if written to text field
                    // TODO: when mouseclicking autocomplete list item, textfield gets cleared because of this
                    //Object item = browserField.getSelectedItem();
                    //browserField.setSelectedItem(item);

\subsection{gui/ProjectExplorerTable.java}
        // TODO: what should be here anyway?
        // this.setPreferredScrollableViewportSize(new Dimension(280, 400));
        // set the right visible columns for table type
                    // TODO: flash selected row red for 100 ms, perhaps? - might require a custom cell renderer
            // TODO: does this look better without the "*"?
            // - not to me, but don't really care anymore :)
            // TODO: some portable way to get a File for disk drive? Or maybe a Setting for export-dirs?
                for (String type : new String[]{"dat", "tdt" /*, "srm" */}) {   // TODO: support for SRM files is missing
                                // TODO: tell somehow, not with popup, if export was successful; statusbar perhaps?

\subsection{gui/ProjectInformationPanel.java}

\subsection{gui/SequenceColumn.java}
        // TODO: remove the need for this hack
            setNumberFormat(new DecimalFormat("0.000E0"));  // TODO: maybe make a new class that could show the format: 1.123e+05

\subsection{gui/SettingsDialog.java}

\subsection{gui/StereoPlot.java}

\subsection{gui/StyledCellEditor.java}

\subsection{gui/StyledTableCellRenderer.java}

\subsection{gui/StyledWrapper.java}
//        public Insets insets = null; // TODO: is this also necessary?

\subsection{squid/Degausser.java}
        //needs to call new functions setDelay() and setRamp(). TODO
        //TODO: do we need to check values? (original does)
        //TODO: problem when Degausser and Magnetometer uses same port :/

\subsection{squid/Handler.java}
        // TODO: maybe currentVelocity is not in steps per second?
        double timeSpent = (System.currentTimeMillis() - estimatedPositionStartTime) / 1000.0;    // in seconds
        // TODO: calculate the acceleration and deceleration corrections
        // TODO: maybe currentVelocity is not in steps per second?
        double timeSpent = (System.currentTimeMillis() - estimatedRotationStartTime) / 1000.0;    // in seconds
        // prevent going over the end limit

\subsection{squid/Magnetometer.java}
            //Original sets range and filter to 1x and disable fast-slew, TODO: check if right, do we need status confirm?
        //when to use flux counting and when not? TODO
        //TODO: problem when Degausser and Magnetometer uses same port :/

\subsection{squid/SerialIO.java}

\subsection{squid/SerialIOEvent.java}

\subsection{squid/SerialIOException.java}

\subsection{squid/SerialIOListener.java}

\subsection{squid/SerialParameters.java}

\subsection{squid/Squid.java}
        // TODO

\subsection{squid/SquidEmulator.java}
            No further development needed. Was only little help on testing and wad dumped soon after testing in laboratorium.

\subsection{squid/SquidFront.java}
	No further development needed. Have been used for testing some commands, but human control is too risky as we noticed.



\section{Improvement suggestions}
\label{sec:improv}

What's left for the program to make it perfect :)

\subsection{...}

\subsection{Porting to linux}
