\subsection{Ikayaki.java}
        //add(main.getStatusBar(), "South");    // TODO: there is no status bar

\subsection{MeasurementEvent.java}

\subsection{MeasurementListener.java}

\subsection{MeasurementResult.java}

\subsection{MeasurementSequence.java}

\subsection{MeasurementStep.java}

\subsection{MeasurementValue.java}
                        // TODO: should there be the full vectors, or normalized vectors (length=1)?

\subsection{Project.java}
    private static final boolean DEBUG = false;      // TODO: used for testing the measurements without a Squid
        // TODO: use a DTD for the document
        // create document's root element
        return false; // TODO
            out.print(pad(s, Math.max(getName().length(), 8), 1));      // TODO: how should this be calculated?
            // strike
            return false;   // TODO: maybe its better to throw an exception, so that it would be easier to find bugs
            return false;    // TODO: maybe its better to throw an exception, so that it would be easier to find bugs
            return false; // TODO: maybe its better to throw an exception, so that it would be easier to find bugs
            return false;   // TODO: maybe its better to throw an exception, so that it would be easier to find bugs
    // TODO: is this comment even close?
                // TODO: these should be in one Magnetometer method
                    // TODO: these should be in one Magnetometer method

\subsection{ProjectEvent.java}

\subsection{ProjectListener.java}

\subsection{Settings.java}

\subsection{gui/AbstractPlot.java}

\subsection{gui/CalibrationPanel.java}

\subsection{gui/ComponentFlasher.java}

\subsection{gui/DeviceSettingsPanel.java}

There are many settings in Squid that apparently are not needed. So there is now only those which seems only to be needed.

There is no check if all systems have same port, only degausser and magnetometer can have same port.

Window cannot be closed with ESC-key as we wanted.

\subsection{gui/FittedComboBoxRenderer.java}

\subsection{gui/GenericFileFilter.java}

\subsection{gui/IntensityPlot.java}
        // TODO draw ticks and numbers for y-axis
        // x-axis ticks
        // TODO draw ticks and numbers for x-axis
        // y-axis unit
        // x-axis unit

\subsection{gui/MagnetometerStatusPanel.java}

If any two handler positions (read from Settings) are the same, the corresponding move-radiobuttons won't be moved correctly, as they are dumped into a TreeMap. There are now same positions in current lab setup, and don't know if there ever could be, so this won't matter; however, Device Configuration doesn't check what is inputted there, so let's just hope noone messes with those.

Call for updateButtonPositions(...) is in paintComponent, as otherwise the button positions won't stick; don't know what's the right place then for that method call.

Changing the "Measure BG"/"Measure XYZ" button text changes whole MagnetometerStatusPanel width; should prevent that somehow.

\subsection{gui/MainMenuBar.java}
            //exportProjectMenu.add(exportProjectToSRM);    // TODO: support for SRM files is missing

\subsection{gui/MainStatusBar.java}
        return; // TODO
        return; // TODO
        return; // TODO

\subsection{gui/MainViewPanel.java}
                    // TODO: what should be done now? give error message?
                    //e.printStackTrace();
        // TODO: make this as an Action?

\subsection{gui/MeasurementControlsPanel.java}

\subsection{gui/MeasurementDetailsPanel.java}

\subsection{gui/MeasurementGraphsPanel.java}
        return; // TODO

\subsection{gui/MeasurementSequencePanel.java}
        // TODO: Put these same actions to the program's main menu. Each action might then need to find out the selected rows itself and monitor the ListSelectionModel.

\subsection{gui/MeasurementSequenceTableModel.java}

\subsection{gui/NullableDecimalFormat.java}

\subsection{gui/Plot.java}

\subsection{gui/PositiveDecimalFormat.java}

\subsection{gui/PrintPanel.java}

\subsection{gui/ProgramSettingsPanel.java}

\subsection{gui/ProjectComponent.java}


\subsection{gui/ProjectExplorerPanel.java}

Cycling through popup menu list with up/down keys changes directory; it shouldn't, but can't recognize those changed-events from mouse clicks.

When mouseclicking autocomplete list item, textfield gets cleared because of setBrowserFieldPopup(...); no easy way around this, as other ways tried cause more other problems.

Many problems here arise from the fact that JComboBox isn't designed for recklessly changing popup menu contents (as we swap between directory history and autocomplete results). It might have been better to make an own component here, instead of using JComboBox, as now the whole thing has a lot of bubblegun stitching. But, it's good enough now.


\subsection{gui/ProjectExplorerTable.java}

Table columns have no indication for sort column, as Esko didn't like the *-indicator :)
	    
ProjectExplorerPopupMenu uses new File("A:/") for disk drive; should be changed for any linux/etc porting.

SRM export is commented out, as it's not supported (in Project).

There are no messages telling if exporting was succesful or not (as I didn't want any popups for it)... Also, exporting overwrites any previous files with the same name; this might be just what the user wants, but it could also cause an unhappy surprise.


\subsection{gui/ProjectInformationPanel.java}

\subsection{gui/SequenceColumn.java}
        // TODO: remove the need for this hack
            setNumberFormat(new DecimalFormat("0.000E0"));  // TODO: maybe make a new class that could show the format: 1.123e+05

\subsection{gui/SettingsDialog.java}

\subsection{gui/StereoPlot.java}

\subsection{gui/StyledCellEditor.java}

\subsection{gui/StyledTableCellRenderer.java}

\subsection{gui/StyledWrapper.java}
//        public Insets insets = null; // TODO: is this also necessary?

\subsection{squid/Degausser.java}

This class seems to work alright, but there is some issues which are uncertain. It have not been tested so that Degausser and Magnetometer are in same port, more likely there will be many problems. This class trusts that everything goes alright, would be good to confirm status sometimes and check that everything is going well. ExecuteRampUp() and executeRampDown() are not used because there is risk that ramp stays up, executeRampCycle is only used.

\subsection{squid/Handler.java}

Estimated movement does not calculate acceleration so its not exactly correct, but works well enough.

\subsection{squid/Magnetometer.java}

This class seems to work alright, but there is some issues which are uncertain. It have not been tested so that Degausser and Magnetometer are in same port, more likely there will be many problems. We only use filter 1x and range 1x and disable fast-slew, no idea if other options are needed for these. And we do not check status from magnetometer, we just hope all goes well. In measuring we do not Never use flux counting, never be needed tough.

\subsection{squid/SerialIO.java}

\subsection{squid/SerialIOEvent.java}

\subsection{squid/SerialIOException.java}

\subsection{squid/SerialIOListener.java}

\subsection{squid/SerialParameters.java}

\subsection{squid/Squid.java}

There is no functionality for UpdateSettings(), so when settings are changed program needs to be restarted. This should be corrected as it was planned. And there is no call from DeviceSettings for this.

\subsection{squid/SquidEmulator.java}

No further development needed. Was only little help on testing and wad dumped soon after testing in laboratorium.

\subsection{squid/SquidFront.java}

No further development needed. Have been used for testing some commands, but human control is too risky as we noticed.
