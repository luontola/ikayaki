\subsection{Ikayaki.java}
        //add(main.getStatusBar(), "South");    // TODO: there is no status bar

\subsection{MeasurementEvent.java}

\subsection{MeasurementListener.java}

\subsection{MeasurementResult.java}

\subsection{MeasurementSequence.java}

\subsection{MeasurementStep.java}

\subsection{MeasurementValue.java}
                        // TODO: should there be the full vectors, or normalized vectors (length=1)?

\subsection{Project.java}
    private static final boolean DEBUG = false;      // TODO: used for testing the measurements without a Squid
        // TODO: use a DTD for the document
        // create document's root element
        return false; // TODO
            out.print(pad(s, Math.max(getName().length(), 8), 1));      // TODO: how should this be calculated?
            // strike
            return false;   // TODO: maybe its better to throw an exception, so that it would be easier to find bugs
            return false;    // TODO: maybe its better to throw an exception, so that it would be easier to find bugs
            return false; // TODO: maybe its better to throw an exception, so that it would be easier to find bugs
            return false;   // TODO: maybe its better to throw an exception, so that it would be easier to find bugs
    // TODO: is this comment even close?
                // TODO: these should be in one Magnetometer method
                    // TODO: these should be in one Magnetometer method

\subsection{ProjectEvent.java}

\subsection{ProjectListener.java}

\subsection{Settings.java}

\subsection{gui/AbstractPlot.java}

\subsection{gui/CalibrationPanel.java}

\subsection{gui/ComponentFlasher.java}

\subsection{gui/DeviceSettingsPanel.java}

There are many settings in Squid that apparently are not needed. So there is now only those which seems only to be needed.

There is no check if all systems have same port, only degausser and magnetometer can have same port.
            // TODO: doesn't work

\subsection{gui/FittedComboBoxRenderer.java}

\subsection{gui/GenericFileFilter.java}

\subsection{gui/IntensityPlot.java}
        // TODO draw ticks and numbers for y-axis
        // x-axis ticks
        // TODO draw ticks and numbers for x-axis
        // y-axis unit
        // x-axis unit

\subsection{gui/MagnetometerStatusPanel.java}
    // TODO: some way to get the actual max-position?
    // handler positions, read from Settings
    // WARNING: all of these must differ or we have trouble...
        // TODO: WARNING: if two positions are the same, previous one gets replaced
        // TODO: only need to call updateButtonPositions here, but it won't work so now it's called
        // every time in paintComponent
        // TODO: is this needed?
        // TODO: what would be the right place for this call?
        // let Swing erase the background
                // TODO: this wouldn't actually be needed if we get MeasurementEvent every time
                // handler stops, but let's just make sure :)
                // TODO: different speed in measurement zone
            // TODO: as this is the biggest button is measurePanel, changing its text changes whole panel width,
            // and, in the end, whole MagnetometerStatusPanel width; should prevent that from happening

\subsection{gui/MainMenuBar.java}
            //exportProjectMenu.add(exportProjectToSRM);    // TODO: support for SRM files is missing

\subsection{gui/MainStatusBar.java}
        return; // TODO
        return; // TODO
        return; // TODO

\subsection{gui/MainViewPanel.java}
                    // TODO: what should be done now? give error message?
                    //e.printStackTrace();
        // TODO: make this as an Action?

\subsection{gui/MeasurementControlsPanel.java}

\subsection{gui/MeasurementDetailsPanel.java}

\subsection{gui/MeasurementGraphsPanel.java}
        return; // TODO

\subsection{gui/MeasurementSequencePanel.java}
        // TODO: Put these same actions to the program's main menu. Each action might then need to find out the selected rows itself and monitor the ListSelectionModel.

\subsection{gui/MeasurementSequenceTableModel.java}

\subsection{gui/NullableDecimalFormat.java}

\subsection{gui/Plot.java}

\subsection{gui/PositiveDecimalFormat.java}

\subsection{gui/PrintPanel.java}

\subsection{gui/ProgramSettingsPanel.java}

\subsection{gui/ProjectComponent.java}


\subsection{gui/ProjectExplorerPanel.java}

Cycling through popup menu list with up/down keys changes directory; it shouldn't, but can't recognize those changed-events from mouse clicks.

When mouseclicking autocomplete list item, textfield gets cleared because of setBrowserFieldPopup(...); no easy way around this, as other ways tried cause more other problems.

Many problems here arise from the fact that JComboBox isn't designed for recklessly changing popup menu contents (as we swap between directory history and autocomplete results). It might have been better to make an own component here, instead of using JComboBox, as now the whole thing has a lot of bubblegun stitching. But, it's good enough now.


\subsection{gui/ProjectExplorerTable.java}

Table columns have no indication for sort column, as Esko didn't like the *-indicator :)
	    
ProjectExplorerPopupMenu uses new File("A:/") for disk drive; should be changed for any linux/etc porting.

SRM export is commented out, as it's not supported (in Project).

There are no messages telling if exporting was succesful or not (as I didn't want any popups for it)... Also, exporting overwrites any previous files with the same name; this might be just what the user wants, but it could also cause an unhappy surprise.


\subsection{gui/ProjectInformationPanel.java}

\subsection{gui/SequenceColumn.java}
        // TODO: remove the need for this hack
            setNumberFormat(new DecimalFormat("0.000E0"));  // TODO: maybe make a new class that could show the format: 1.123e+05

\subsection{gui/SettingsDialog.java}

\subsection{gui/StereoPlot.java}

\subsection{gui/StyledCellEditor.java}

\subsection{gui/StyledTableCellRenderer.java}

\subsection{gui/StyledWrapper.java}
//        public Insets insets = null; // TODO: is this also necessary?

\subsection{squid/Degausser.java}
        //needs to call new functions setDelay() and setRamp(). TODO
        //TODO: do we need to check values? (original does)
        //TODO: problem when Degausser and Magnetometer uses same port :/

\subsection{squid/Handler.java}
        // TODO: maybe currentVelocity is not in steps per second?
        double timeSpent = (System.currentTimeMillis() - estimatedPositionStartTime) / 1000.0;    // in seconds
        // TODO: calculate the acceleration and deceleration corrections
        // TODO: maybe currentVelocity is not in steps per second?
        double timeSpent = (System.currentTimeMillis() - estimatedRotationStartTime) / 1000.0;    // in seconds
        // prevent going over the end limit

\subsection{squid/Magnetometer.java}
            //Original sets range and filter to 1x and disable fast-slew, TODO: check if right, do we need status confirm?
        //when to use flux counting and when not? TODO
        //TODO: problem when Degausser and Magnetometer uses same port :/

\subsection{squid/SerialIO.java}

\subsection{squid/SerialIOEvent.java}

\subsection{squid/SerialIOException.java}

\subsection{squid/SerialIOListener.java}

\subsection{squid/SerialParameters.java}

\subsection{squid/Squid.java}
        // TODO

\subsection{squid/SquidEmulator.java}
            No further development needed. Was only little help on testing and wad dumped soon after testing in laboratorium.

\subsection{squid/SquidFront.java}
	No further development needed. Have been used for testing some commands, but human control is too risky as we noticed.
