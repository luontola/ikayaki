\beginClass{MeasurementGraphsPanel}
\classComment{}
\classPackage{ikayaki.gui}
\classDeclaration{public class MeasurementGraphsPanel}
\classExtends{ProjectComponent}
\classImplements{MeasurementListener}
\classCreatedBy{MainViewPanel}
\classUses{MeasurementStep}
\closeClass


\beginClass{GeneralPlot}
\classComment{abstract class that implements general construction of a graphical plot}
\classPackage{ikayaki.gui}
\classDeclaration{public abstract GeneralPlot}
\classExtends{JPanel}
\classUses{Measurement}
\classSubclass{StereoPlot}
\classSubclass{IntensityPlot}
\closeClass

\beginField{measurements}
\fieldDeclaration{private Vector<Measurement> measurement}
\fieldValue{null}
\fieldComment{Contains all the data that is shown in this graph.}
\closeField

\beginMethod{addMeasurement}
\methodDeclaration{public void addMeasurement(int declination, int inclination)}
\methodComment{Adds new measurement data to plot.}
\methodParam{declination}{Declination coordinate of the measurement.}
\methodParam{inclination}{Inclination coordinate of the measurement.}
\closeMethod


\beginMethod{highlightMeasurement}
\methodDeclaration{public void highlightMeasurement(int index)}
\methodComment{High lights measurement in plot in the given index.}
\methodParam{index}{Index of measurement to be highlighted.}
\methodThrows{IndexOutOfBoundsException}{If no such measurement existed.}
\closeMethod

\beginMethod{highlightMeasurementRange}
\methodDeclaration{public void highLightMeasurementRange(int from, int to)}
\methodComment{Highlights a set of measurements in given range.}
\methodParam{from}{Starting index of highlighted measurements.}
\methodParam{to}{End index of highlighted meeasurements.}
\methodThrows{IndexOutOfBoundsException}{If one or both of the indices are out of bounds.}
\closeMethod

\beginMethod{unHighlightAll}
\methodDeclaration{public void unHighlightAll()}
\methodComment{dehighlights all values in this graph.}
\closeMethod

\beginMethod{resetGraph}
\methodDeclaration{public void resetGraph()}
\methodComment{Removes all measurements from the graph.}
\closeMethod

\beginMethod{getNumMeasurements}
\methodDeclaration{public int getNumMeasurements()}
\methodComment{Returns the number of measurements in this graph.}
\methodReturn{Number of measurements.}
\closeMethod

\beginClass{IntensityPlot}
\classComment{Implements intensity graph plot.}
\classPackage{ikayaki.gui}
\classDeclaration{public class IntensityPlot}
\classExtends{GeneralPlot}
\classCreatedBy{MeasurementGraphsPanel}
\closeClass

\beginClass{StereoPlot}
\classComment{Implements stereo graph plot.}
\classPackage{ikayaki.gui}
\classDeclaration{public class IntensityPlot}
\classExtends{GeneralPlot}
\classCreatedBy{MeasurementGraphsPanel}
\closeClass