\beginClass{MainStatusBar}
\classPackage{ikayaki.gui}
\classDeclaration{public class MainStatusBar}
\classExtends{ProjectComponent}
\classCreatedBy{MainViewPanel}
\classComment{Creates its components and listens project events on status change and calculates estimated time for measurement}
\classEvent{On Measurement Event}{recalculates progress and updates status for current measurement}
\closeClass

\beginField{measurementStatus}
\fieldDeclaration{private JLabel measurementStatus}
\fieldComment{text comment of current status(moving,measurement,demagnetization)}
\closeField

\beginField{measurementProgress}
\fieldDeclaration{private JProgressBar measurementProgress}
\fieldComment{progress of sequence/measurement as per cent of whole process}
\closeField

\beginField{currentSequence}
\fieldDeclaration{private int[] currentSequence}
\fieldComment{current projects sequence}
\closeField

\beginField{projectType}
\fieldDeclaration{private int projectType}
\fieldComment{current projects type (we know if we are doing demagnetization or not)}
\closeField

\beginMethod{MainStatusBar()}
\methodDeclaration{public MainStatusBar()}
\methodComment{
    Creates all components with default settings and sets Listener for MeasurementEvent.
}
\closeMethod

\beginMethod{calculateStatus(int,int)}
\methodDeclaration{private void calculateStatus(String phase, int sequenceStep, int currentStep)}
\methodComment{
    Recalculates current progress and updates status.
}
\closeMethod

\beginMethod{setMeasurement(int,int[])}
\methodDeclaration{private void setMeasurement(int projectType, int[] sequence)}
\methodComment{
    Formats status and creates new measurement status values.
}
\closeMethod
