\beginClass{ProjectComponent}
\classPackage{ikayaki.gui}
\classDeclaration{public class ProjectComponent}
\classExtends{JPanel}
\classCreatedBy{MainViewPanel}
\classUses{Project}
\classSubclass{ProjectInformationPanel}
\classSubclass{MeasurementSequencePanel}
\classSubclass{MeasurementDetailsPanel}
\classSubclass{MeasurementControlsPanel}
\classSubclass{MeasurementGraphsPanel}
\classSubclass{ProjectExplorerPanel}
\classSubclass{CalibrationPanel}
\classComment{Generic gui component which uses Project and listens MeasurementEvents and ProjectEvents.}
\closeClass

\beginField{project}
\fieldDeclaration{private Project project}
\fieldComment{Active project.}
\closeField

\beginMethod{ProjectComponent()}
\methodDeclaration{public ProjectComponent()}
\methodComment{Initializes this ProjectComponent with no Project (one probably arrives shortly with setProject(Project)).}
\closeMethod

\beginMethod{getProject()}
\methodDeclaration{public Project getProject()}
\methodReturn{this.project.}
\closeMethod

\beginMethod{setProject(Project)}
\methodDeclaration{public void setProject(Project project)}
\methodComment{
	Sets the new project for this ProjectComponent, unregisters MeasurementListener and ProjectListener from old project, saves new Project to project, registers MeasurementListener and ProjectListener to new project.
}
\methodParam{project}{new active project.}
\closeMethod
