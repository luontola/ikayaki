\beginClass{ProjectExplorerPanel}
\classComment{
	Creates a history/autocomplete field (browserField) for choosing the project directory, a listing of project files in that directory (explorerTable) and in that listing a line for creating new project, which has a textbox for project name, an AF/TH ComboBox and a "Create new" button (createNewProjectButton) for actuating the creation. Also has a right-click popup menu for exporting project files.
}
\classPackage{ikayaki.gui}
\classDeclaration{public class ProjectExplorerPanel}
\classExtends{ProjectComponent}
\classCreatedBy{MainViewPanel}
\classUses{MainViewPanel}
\classUses{ProjectExplorerTable}
\classUses{ProjectExplorerPopupMenu}
\classUses{NewProjectPanel}
\classUses{RunQueue}
\classEvent{On browserField change}{send autocomplete-results-finder with browserField's text to RunQueue via runQueue.offer(Runnable), which schedules disk access and displaying autocomplete results in browserField's popup window.}
\classEvent{On browserField down-arrow-click}{show directory history in browserField's popup window.}
\classEvent{On browserField popup window click}{set clicked line as {\it directory}, update {\it files} listing, update explorerTable and browserField.}
\classEvent{On browseButton click}{open a FileChooser dialog for choosing new directory, set it to {\it directory}, update {\it files} listing, update explorerTable and browserField.}
\classEvent{On ProjectEvent}{hilight project whose measuring started, or unhilight one whose measuring ended.}
\closeClass

\beginField{browserField}
\fieldDeclaration{private JComboBox browserField}
\fieldComment{Text field for writing directory to change to. Autocomplete results appear to Combo Box' popup window, scheduled by RunQueue. Directory history appears to the same popup window when the down-arrow right to text field is clicked.}
\closeField

\beginField{browseButton}
\fieldDeclaration{private JButton browseButton}
\closeField

\beginField{explorerTable}
\fieldDeclaration{private ProjectExplorerTable explorerTable}
\closeField

\beginField{newProjectPanel}
\fieldDeclaration{private NewProjectPanel newProjectPanel}
\closeField

\beginField{autocompleteQueue}
\fieldDeclaration{private RunQueue autocompleteQueue}
\fieldValue{new RunQueue(100, true)}
\fieldComment{RunQueue for scheduling autocomplete results to separate thread (disk access and displaying).}
\closeField

\beginField{directory}
\fieldDeclaration{private File directory}
\fieldValue{null}
\fieldComment{Currently open directory.}
\closeField

\beginField{files}
\fieldDeclaration{private Vector<File> files}
\fieldValue{new Vector<File>()}
\fieldComment{Project files in current directory.}
\closeField

\beginMethod{ProjectExplorerPanel(Project)}
\methodDeclaration{public ProjectExplorerPanel(Project project)}
\methodComment{Creates all components, sets {\it directory} as the last open directory, initializes {\it files} with files from that directory.}
\closeMethod

\beginMethod{setProject(Project)}
\methodDeclaration{public void setProject(Project project)}
\methodComment{Call super.setProject(project), hilight selected project, or unhilight unselected project.}
\closeMethod


\beginClass{NewProjectPanel}
\classComment{
	Panel with components for creating a new project. This Panel will be somewhere below the project file listing...
}
\classPackage{ikayaki.gui}
\classDeclaration{public class NewProjectPanel}
\classExtends{JPanel}
\classCreatedBy{ProjectExplorerPanel}
\classUses{MainViewPanel}
\classEvent{On createNewProjectButton click}{call Project.createXXXProject(File) with filename from newProjectField; if returns null, show error message and do nothing. Otherwise, update file listing, set new project active, tell explorerTable to reset newProjectField and newProjectType and call MainViewPanel.changeProject(Project) with returned Project.}
\closeClass

\beginField{newProjectField}
\fieldDeclaration{private JTextField newProject}
\closeField

\beginField{JComboBox newProjectType}
\fieldDeclaration{private JComboBox newProjectType}
\fieldValue{AF/Thellier/Thermal}
\closeField

\beginField{createNewProjectButton}
\fieldDeclaration{private JButton createNewProjectButton}
\closeField


\beginClass{ProjectExplorerTable}
\classComment{
	Creates a list of project files in {\it directory}. Handles loading selected projects and showing export popup menu (ProjectExplorerPopupMenu). Inner class of ProjectExplorerPanel.
}
% What the hell anyway, I'm confused.. *whacks some stupid monsters with a chainsaw* NNNNnnnnnnnnrrnnnnnnnn-splat-grah-blurts-blorp-stuff-guttering
\classPackage{ikayaki.gui}
\classDeclaration{public class ProjectExplorerTable}
\classExtends{JTable}
\classCreatedBy{ProjectExplorerPanel}
\classUses{MainViewPanel}
\classEvent{On explorerTable click}{call Project.loadProject(File) with clicked project file, call MainViewPanel.changeProject(Project) with returned Project unless null, on which case show error message and revert explorerTable selection to old project, if any.}
\classEvent{On explorerTable mouse right-click}{create a ProjectExplorerPopupMenu for right-clicked project file.}
\closeClass

\beginField{projectExplorerTableModel}
\fieldDeclaration{private TableModel projectExplorerTableModel}
\fieldComment{TableModel which handles data from files (in upper-class ProjectExplorerPanel). Unnamed inner class.}
\closeField


\beginClass{ProjectExplorerPopupMenu}
\classComment{
	Shows popup menu with export choices: AF (.dat), Thellier (.tdt) and Thermal (.tdt), and executes selected.
}
\classPackage{ikayaki.gui}
\classDeclaration{public class ProjectExplorerPopupMenu}
\classExtends{JPopupMenu}
\classCreatedBy{ProjectExplorerPanel}
\classEvent{On selectItem mouseEvent}{call project.exportXXX(); if false is returned, show error message.}
\closeClass
