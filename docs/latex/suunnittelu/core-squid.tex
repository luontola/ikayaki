\beginClass{Squid}
\classComment{
	Offers an interface for controlling the SQUID system. Reads settings from the Settings class. Creates instances of the degausser, handler and magnetometer interfaces and offers handlers for them.
}
\classPackage{ikayaki.squid}
\classDeclaration{public class Squid}
\classCreatedBy{MainViewPanel}
\classUses{Settings}
\classUses{Degausser}
\classUses{Handler}
\classUses{Magnetometer}
\classPatterns{Singleton}
\closeClass

\beginField{instance}
\fieldDeclaration{private static final Squid instance}
\fieldComment{Instance of the Squid interface.}
\closeField

\beginField{owner}
\fieldDeclaration{private Project owner}
\fieldComment{The project that is currently using the Squid, or null if no project is using it.}
\closeField

\beginField{degausser}
\fieldDeclaration{private final Degausser degausser}
\fieldComment{Instance of the degausser interface.}
\closeField

\beginField{handler}
\fieldDeclaration{private final Handler handler}
\fieldComment{Instance of the handler interface.}
\closeField

\beginField{magnetometer}
\fieldDeclaration{private final Magnetometer magnetometer}
\fieldComment{Instance of the magnetometer interface.}
\closeField

\beginMethod{instance()}
\methodDeclaration{public static synchronized Settings instance()}
\methodComment{
	Returns a reference to the Squid. If it has not yet been created, will create one.
}
\closeMethod

\beginMethod{Squid()}
\methodDeclaration{private Squid()}
\methodComment{
    Initializes the Squid interface. Creates instances of Degausser, Handler and Magnetometer.
}
\closeMethod

\beginMethod{getDegausser()}
\methodDeclaration{public Degausser getDegausser()}
\methodComment{
    Returns an interface for controlling the degausser.
}
\methodReturn{Handler for Degausser if available}
\closeMethod

\beginMethod{getHandler()}
\methodDeclaration{public Handler getHandler()}
\methodComment{
    Returns an interface for controlling the handler.
}
\methodReturn{Handler for Handler if available}
\closeMethod

\beginMethod{getMagnetometer()}
\methodDeclaration{public Magnetometer getMagnetometer()}
\methodComment{
    Returns an interface for controlling the magnetometer.
}
\methodReturn{Handler for Magnetometer if available}
\closeMethod

\beginMethod{updateSettings()}
\methodDeclaration{public synchronized void updateSettings()}
\methodComment{
    Reloads all settings and initializes all the device interfaces. This method should be called when changes are made to the device parameters.
}
\closeMethod

\beginMethod{isOK()}
\methodDeclaration{public synchronized boolean isOK()}
\methodComment{
    Checks whether all devices are working correctly.
}
\methodReturn{true if everything is correct, otherwise false.}
\closeMethod

\beginMethod{setOwner(Project owner)}
\methodDeclaration{public synchronized boolean setOwner(Project owner)}
\methodComment{
	Sets the owner of the Squid. Only one project may have access to the Squid at a time. This method may be called only from the Project class.
}
\methodParam{owner}{the project that will have exclusive access to the Squid.}
\methodReturn{true if successful, false if the existing owner had a running measurement.}
\closeMethod

\beginMethod{getOwner()}
\methodDeclaration{public synchronized Project getOwner()}
\methodComment{
    Returns project that is currently using the Squid.
}
\methodReturn{the project, or null if none is using the Squid.}
\closeMethod

\beginClass{Degausser}
\classComment{
    Offers an interface for controlling the degausser (demagnetizer). Because the data link is implemented in the degausser by a single board computer running a small basic program, the response time of the degausser to commands is slow. This class will make sure that would calls are not made faster than the device can handle.
}
\classPackage{ikayaki.squid}
\classDeclaration{public class Degausser}
\classImplements{SerialIOListener}
\classCreatedBy{Squid}
\classUses{Settings}
\classUses{SerialIO}
\classEvent{On new IO message}{reads the message and puts it in a buffer}
\closeClass

\beginField{messageBuffer}
\fieldDeclaration{private Stack messageBuffer}
\fieldComment{buffer for incoming messages, readed when needed.}
\closeField

\beginField{status}
\fieldDeclaration{private String status}
\fieldComment{Degaussers current status}
\closeField

\beginField{comPort}
\fieldDeclaration{private String comPort}
\fieldComment{COM port for communication}
\closeField

\beginField{degausserCoil}
\fieldDeclaration{private int degausserCoil}
\fieldComment{(X, Y, Z) = (0,1,2) default axis Z}
\closeField

\beginField{degausserAmplitude}
\fieldDeclaration{private int degausserAmplitude}
\fieldComment{0->3000 default amp 0}
\closeField

\beginField{degausserDelay}
\fieldDeclaration{private int degausserDelay}
\fieldComment{1-9 seconds default delay 1 second}
\closeField

\beginField{degausserRamp}
\fieldDeclaration{private int degausserRamp}
\fieldComment{(3, 5, 7, 9) default 3}
\closeField

\beginField{degausserStatus}
\fieldDeclaration{private char degausserRamp}
\fieldComment{Z=Zero, T=Tracking, ?=Unknown}
\closeField

\beginMethod{Degausser()}
\methodDeclaration{public Degausser()}
\methodComment{
    Creates a new degausser interface. Opens connection to degausser COM port (if not open yet) and reads settings from the Setting class.
}
\closeMethod

\beginMethod{updateSettings()}
\methodDeclaration{public void updateSettings()}
\methodComment{
    Reloads all settings and initializes the degausser interface. This method should be called when changes are made to the device parameters.
}
\closeMethod

\beginMethod{setCoil(char)}
\methodDeclaration{public void setCoil(char coil)}
\methodComment{
    Sets coil X,Y,Z.
}
\methodParam{coil}{coil to set on.}
\closeMethod

\beginMethod{setAmplitude(int)}
\methodDeclaration{public void setAmplitude(int amplitude)}
\methodComment{
    Sets amplitude to ramp, range 0 to 3000.
}
\methodParam{amplitude}{amplitude to demag.}
\closeMethod

\beginMethod{executeRampUp()}
\methodDeclaration{public void executeRampUp()}
\methodComment{
    Performs Ramp up.
}
\closeMethod

\beginMethod{executeRampDown()}
\methodDeclaration{public void executeRampDown()}
\methodComment{
    Brings Ramp down.
}
\closeMethod

\beginMethod{executeRampCycle()}
\methodDeclaration{public void executeRampCycle()}
\methodComment{
    Performs Ramp up and down.
}
\closeMethod

\beginMethod{demagnetize(int amplitude)}
\methodDeclaration{public boolean demagnetize(int amplitude)}
\methodComment{
    Performs full sequence to demagnetize with given amplitude. Blocking method.
}
\methodParam{amplitude}{amplitude to demag.}
\methodReturn{true if process was sended succesfully, otherwise false.}
\closeMethod

\beginMethod{getStatus()}
\methodDeclaration{public String getStatus()}
\methodComment{
    Gives configuration and ramp status.
}
\closeMethod

\beginMethod{isOK()}
\methodDeclaration{public boolean isOK()}
\methodComment{
    checks if connection is ok.
}
\methodReturn{True if ok}
\closeMethod


\beginClass{Handler}
\classComment{
    Offers an interface for controlling the sample handler.
}
\classPackage{ikayaki.squid}
\classDeclaration{public class Handler}
\classImplements{SerialIOListener}
\classCreatedBy{Squid}
\classUses{Settings}
\classUses{SerialIO}
\classEvent{On New IO Message }{reads message and puts it in Buffer}
\closeClass

\beginField{messageBuffer}
\fieldDeclaration{private Stack messageBuffer}
\fieldComment{buffer for incoming messages, readed when needed.}
\closeField

\beginField{status}
\fieldDeclaration{private String status}
\fieldComment{Handlers current status}
\closeField

\beginField{comPort}
\fieldDeclaration{private String comPort}
\fieldComment{COM port for communication}
\closeField

\beginField{acceleration}
\fieldDeclaration{private int acceleration}
\fieldComment{value between 0 and 127 default 5. Settings in the
20-50 range are usually employed.}
\closeField

\beginField{deceleration}
\fieldDeclaration{private int deceleration}
\fieldComment{value between 0 and 127 default 10. Settings in the
20-50 range are usually employed.}
\closeField

\beginField{velocity}
\fieldDeclaration{private int velocity}
\fieldComment{value between 50 and 12 000. The
decimal number issued is 10 times the actual pulse rate to the motor. Since the
motor requires 200 pulses (full step) or 400 pulses (half step) per revolution,
a speed setting of M10000 sets the motor to revolve at 5 revolutions per second
in full step or 2.5 revolutions in half step. This rate is one-half the sample
rate rotation due to the pulley ratios. The sample handler is set up at the
factory for half stepping.}
\closeField

\beginField{measurementVelocity}
\fieldDeclaration{private int measurementVelocity}
\fieldComment{speed in measurement, should be small}
\closeField

\beginField{handlerStatus}
\fieldDeclaration{private String handlerStatus}
\fieldComment{5 end of move, previous G command complete, 7 hard limit stop, G motor is currently indexing}
\closeField

\beginField{currentPosition}
\fieldDeclaration{private int currentPosition}
\fieldComment{value between 1 and 16,777,215}
\closeField

\beginField{homePosition}
\fieldDeclaration{private int homePosition}
\fieldComment{value between 1 and 16,777,215}
\closeField

\beginField{transverseYAFPosition}
\fieldDeclaration{private int transverseYAFPosition}
\fieldComment{AF demag position for transverse}
\closeField

\beginField{axialAFPosition}
\fieldDeclaration{private int axialAFPosition}
\fieldComment{axial AF demag position in steps, must be divisible by 10. Relative to Home.}
\closeField

\beginField{backgroundPosition}
\fieldDeclaration{private int backgroundPosition}
\fieldComment{Position in steps, must be divisible by 10. Relative to Home.}
\closeField

\beginField{measurementPosition}
\fieldDeclaration{private int measurementPosition}
\fieldComment{Position in steps, must be divisible by 10. Relative to Home.}
\closeField

\beginField{currentRotation}
\fieldDeclaration{private int currentRotation}
\fieldComment{angles are between 0 (0) and 2000 (360)}
\closeField

\beginMethod{Handler()}
\methodDeclaration{public Handler()}
\methodComment{
    Creates a new handler interface. Opens connection to handler COM port and reads settings from the Setting class.
}
\closeMethod

\beginMethod{updateSettings()}
\methodDeclaration{public void updateSettings()}
\methodComment{
    Reloads all settings and initializes the handler interface. This method should be called when changes are made to the device parameters.
}
\closeMethod

\beginMethod{getStatus()}
\methodDeclaration{public String getStatus()}
\methodComment{
    Returns current status on Sample Handler.
}
\closeMethod

\beginMethod{isOK()}
\methodDeclaration{public boolean isOK()}
\methodComment{
    checks if connection is ok.
}
\methodReturn{True if ok}
\closeMethod

\beginMethod{moveToHome()}
\methodDeclaration{public void moveToHome()}
\methodComment{
    Send command to sample holder to move home
}
\closeMethod

\beginMethod{moveToDegausser()}
\methodDeclaration{public void moveToDegausser()}
\methodComment{
    Send handler to Degauss position
}
\closeMethod

\beginMethod{moveToMeasurement()}
\methodDeclaration{public void moveToMeasurement()}
\methodComment{
   Send handler to Measure position
}
\closeMethod

\beginMethod{moveToBackground()}
\methodDeclaration{public void moveToBackground()}
\methodComment{
   Send handler to Background position
}
\closeMethod

\beginMethod{moveToPos(int)}
\methodDeclaration{public boolean moveToPos(int pos)}
\methodComment{
    Value must be between 1 and 16,777,215. return true if good pos-value and moves handler there.
}
\methodParam{pos}{Position where handler are sent}
\methodReturn{true if given position was ok, otherwise false.}
\closeMethod

\beginMethod{stop()}
\methodDeclaration{public void stop()}
\methodComment{
    Tells handler to stop its curren job.
}
\closeMethod

\beginMethod{rotateTo(int)}
\methodDeclaration{public void rotateTo(int angle)}
\methodComment{
    Rotates the handler to the specified angle.
}
\methodParam{angle}{the angle in degrees to rotate the handler to.}
\closeMethod

\beginMethod{setOnline()}
\methodDeclaration{private void setOnline()}
\methodComment{
    Sends message to handler go online (@0).
}
\closeMethod

\beginMethod{setAcceleration(int a)}
\methodDeclaration{private void setAcceleration(int a)}
\methodComment{
    Sends message to handler to set acceleration (Aa).
}
\methodParam{a}{Acceleration is a number from 0 to 127}
\closeMethod

\beginMethod{setDeceleration(int d)}
\methodDeclaration{private void setDeceleration(int d)}
\methodComment{
    Sends message to handler to set deceleration (Dd).
}
\methodParam{a}{Deceleration is a number from 0 to 127}
\closeMethod

\beginMethod{setBaseSpeed(int b)}
\methodDeclaration{private void setBaseSpeed(int b)}
\methodComment{
    Sends message to handler to set base speed. Base rate is the speed at which the motor motion starts and stops. (Bb).
}
\methodParam{b}{Base Speed is pulses per second and has a range of 50 to 5000.}
\closeMethod

\beginMethod{setVelocity(int v)}
\methodDeclaration{private void setVelocity(int v)}
\methodComment{
    Sends message to handler to set maximum velocity (Mv).
}
\methodParam{v}{Velocity range is 50 to 20,000}
\closeMethod

\beginMethod{setHoldTime(int h)}
\methodDeclaration{private void setHoldTime(int h)}
\methodComment{
    This command causes the POWER pin to be pulled low within a specified number of ticks after a move of the motor is completed and it will stay low until just prior to the start of the next move command. This command allows the holding torque of the motor to be turned off automatically after a delay for the mechanical system stabilize thus reducing power consumption and allowing the motor to be turned by hand if this feature is required. If the value is zero then the power is left on forever. (CH h).
}
\methodParam{h}{value from 0 to 127 representing the number of clock ticks to leave power on the motor after a move.}
\closeMethod

\beginMethod{setCrystalFrequence(int cf)}
\methodDeclaration{private void setCrystalFrequence(int cf)}
\methodComment{
    numbers. The crystal frequency is used by the chip for setting the base speed and maximum speed and for controlling the time for the wait command. (CX cf).
}
\methodParam{cf}{frequence range is 4,000,000 to 8,000.000}
\closeMethod

\beginMethod{stopExecution()}
\methodDeclaration{private void stopExecution()}
\methodComment{
    This command stops execution of the internal program if it is used in the program. If the motor is indexing it will ramp down and then stop. Use this command to stop the motor after issuing a slew command. (Q).
}
\closeMethod

\beginMethod{performSlew()}
\methodDeclaration{private void performSlew()}
\methodComment{
   Slew the motor up to maximum speed and continue until reaching a hard limit switch or receiving a quit (Q) command. (S).
}
\closeMethod

\beginMethod{setMotorPositive()}
\methodDeclaration{private void setMotorPositive()}
\methodComment{
   Set the motor direction of movement to positive. (+).
}
\closeMethod

\beginMethod{setMotorNegative()}
\methodDeclaration{private void setMotorNegative()}
\methodComment{
   Set the motor direction of movement to negative. (-).
}
\closeMethod

\beginMethod{setSteps(int s)}
\methodDeclaration{private void setSteps(int s)}
\methodComment{
   Set the number of steps to move for the G command. (N s).
}
\methodParam{s}{steps range is 0 to 16,777,215}
\closeMethod

\beginMethod{setPosition(int p)}
\methodDeclaration{private void setPosition(int p)}
\methodComment{
   Set absolut position to move for the G command. (P p).
}
\methodParam{p}{position range is 0 to 16,777,215}
\closeMethod

\beginMethod{go()}
\methodDeclaration{private void go()}
\methodComment{
   Send handler on move (G).
}
\closeMethod

\beginMethod{wait()}
\methodDeclaration{private void wait()}
\methodComment{
   Wait for handler to be idle. Blocking (F). Without the F command the SMC25 will continue to accept commands while the motor is moving. This may be desirable, as when changing speed during a move or working with the inputs or outputs. Or it may be undesirable, such as when you wish to make a series of indexes. Without the F command any subsequent G commands received while the motor is indexing would set the "Not allowed while moving" message. Caution: If this command is used while the motor is executing a Slew command the only way to stop is with a reset or a hard limit switch input.
}
\closeMethod

\beginMethod{verify(char v)}
\methodDeclaration{private String verify(char v)}
\methodComment{
   Gives result for wanted registery. (V v).
}
\methodParam{v}{A Acceleration
B Base speed
D Deceleration
E Internal program
G Steps remaining in current move. Zero if not indexing.
H Hold time
I Input pins
J Slow jog speed
M Maximum speed
N Number of steps to index
O Output pins
P Position. If motor is indexing this returns the position at
the end of the index.
R Internal program pointer used by trace (T) or continue (X)
commands. Also updated by enter (E) command.
W Ticks remaining on wait counter
X Crystal frequency
}
\methodReturn{returns registery as string}
\closeMethod

\beginMethod{setPositionRegister(int r)}
\methodDeclaration{private void setPositionRegister(int r)}
\methodComment{
   Set the position register. This command sets the internal absolute position counter to the value of r. (Z r).
}
\methodParam{r}{position range is 0 to 16,777,215}
\closeMethod

\beginMethod{pollMessage()}
\methodDeclaration{private char pollMessage}
\methodComment{
   Poll the device for any waiting messages such as errors or end of move. (%).
}
\methodReturn{0 Normal, no service required
1 Command error, illegal command sent
2 Range error, an out of range numeric parameter was sent
3 Command invalid while moving (e.g. G, S, H)
4 Command only valid in program (e.g. I, U, L)
5 End of move notice, a previous G command is complete
6 End of wait notice, a previous W command is complete
7 Hard limit stop, the move was stopped by the hard limit
8 End of program notice, internal program has completed
G Motor is indexing and no other notice pending
}
\closeMethod


\beginClass{Magnetometer}
\classComment{
	Offers an interface for controlling the magnetometer.
	
	Commands are at most five characters in length including a carriage return <CR>. The syntax is as follows: "<device><command><subcommand><data><CR>"
}
\classPackage{ikayaki.squid}
\classDeclaration{public class Magnetometer}
\classImplements{SerialIOListener}
\classCreatedBy{Squid}
\classUses{Settings}
\classUses{SerialIO}
\classEvent{On New IO Message }{reads message and puts it in Buffer}
\closeClass

\beginField{messageBuffer}
\fieldDeclaration{private Stack messageBuffer}
\fieldComment{buffer for incoming messages, readed when needed.}
\closeField

\beginField{status}
\fieldDeclaration{private String status}
\fieldComment{Magnetometers current status}
\closeField

\beginField{comPort}
\fieldDeclaration{private String comPort}
\fieldComment{COM port for communication}
\closeField

\beginMethod{Magnetometer()}
\methodDeclaration{public Magnetometer()}
\methodComment{
    Creates a new magnetometer interface. Opens connection to Magnetometer COM port (if its not open already) and reads settings from the Setting class.
}
\closeMethod

\beginMethod{updateSettings()}
\methodDeclaration{public void updateSettings()}
\methodComment{
    Reloads all settings and initializes the magnetometer interface. This method should be called when changes are made to the device parameters.
}
\closeMethod

\beginMethod{reset(char)}
\methodDeclaration{private String reset(char axis)}
\methodComment{
   Reset settings on axis
}
\methodParam{axis}{x,y,x or a (all)}
\closeMethod

\beginMethod{resetCounter(char)}
\methodDeclaration{private void resetCounter(char axis)}
\methodComment{
   Reset counter for axis.
}
\methodParam{axis}{x,y,x or a (all)}
\closeMethod

\beginMethod{configure(char,char,char)}
\methodDeclaration{private void configure(char axis, char subcommand, char option)}
\methodComment{
}
\methodParam{axis}{x,y,x or a (all)}
\methodParam{subcommand}{The CONFIGURE subcommands follow: "F" Set filter configuration. The data subfield sets the filter to the indicated range. The four possible data values are: "1" One Hertz Filter; 1 Hz "T" Ten Hertz Filter; 10 Hz "H" One hundred Hertz Filter; 100 Hz "W" Wide band filter; WB "R" Set DC SQUID electronic range. The data subfield selects the range desired. The four possible data values are: "1" One time range; 1x "T" Ten times range; 10x "H" One hundred times range; 100x "E" Extended range; 1000x "S" Set/Reset the fast-slew option. Two data values are possible: "E" Enable the fast-slew; turn it on. "D" Disable the fast-slew; turn it off. "L" This subcommand opens or closes the SQUID feedback loop or resets the analog signal to +/- 1/2 flux quantum about zero. The three possible data values are: "O" Open the feedback loop. (This command also zeros the flux counter) "C" Close the feedback loop. "P" Pulse-reset (open then close) the feedback loop. (This command also zeros the flux counter)}
\methodParam{option}{see data values from subcommands.}
\closeMethod

\beginMethod{latchAnalog(char)}
\methodDeclaration{private void latchAnalog(char axis)}
\methodComment{
    axis is x,y,x or a (all).
}
\methodParam{axis}{x,y,x or a (all)}
\closeMethod

\beginMethod{latchCounter(char)}
\methodDeclaration{private void latchCounter(char axis)}
\methodComment{
    axis is x,y,x or a (all).
}
\methodParam{axis}{x,y,x or a (all)}
\closeMethod

\beginMethod{getData(char,char,String)}
\methodDeclaration{private String getData(char axis, char command, String datavalues)}
\methodComment{
	Generic send message sender, use with cautios and knowledge. Checks if commands are good.
}
\methodParam{axis}{x,y,x}
\methodParam{command}{"D" Send back the analog data last captured with the LATCH command. The <data> field is not required. "C" Send back the counter value last captured with the LATCH command. The <data> field is not required. "S" Send back status. Various pieces of status can be sent by the magnetometer electronics.}
\methodParam{axis}{Datavalues one or more: "A" Send back all status. "F" Send back all filter status. "R" Send back all range status. "S" Send back slew status. "L" Send back SQUID feedback loop status. Return feedback, waiting time?}
\closeMethod

\beginMethod{measure(char)}
\methodDeclaration{public String measure(char axis)}
\methodComment{
    Performs full sequence to measure axis.
}
\methodParam{axis}{x,y,x.}
\methodReturn{Returns results for measuring.}
\closeMethod

\beginMethod{getStatus()}
\methodDeclaration{public String getStatus()}
\methodComment{
    Returns current status on Sample Handler.
}
\closeMethod

\beginMethod{isOK()}
\methodDeclaration{public boolean isOK()}
\methodComment{
    checks if connection is ok.
}
\methodReturn{True if ok}
\closeMethod
