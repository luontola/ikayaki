
\beginClass{RunQueue}
\classComment{
	Executes Runnable objects in a private worker thread after a pre-defined delay. The worker thread will terminate automatically when there are no runnables to be executed. Optionally executes only the last inserted runnable. All operations are thread-safe.

	This class can be used for example in connection with a "continuous search" invoked by a series of GUI events (such as a DocumentListener), but it is necessary to react to only the last event after a short period of user inactivity.
}
\classPackage{ikayaki.util}
\classDeclaration{public class RunQueue}
\classUses{RunQueue.RunQueueThread}
\classUses{RunQueue.RunDelayed}
\classPatterns{Command}
\closeClass

\beginField{delayMillis}
\fieldDeclaration{private int delayMillis}
\fieldValue{0}
\fieldComment{
	Defines how long is the delay in milliseconds, after which the events need to be run.
}
\closeField

\beginField{execOnlyLast}
\fieldDeclaration{private boolean execOnlyLast}
\fieldValue{false}
\fieldComment{
	Defines if only the last event should be executed. If false, then all of the events are executed in the order of appearance.
}
\closeField

\beginField{queue}
\fieldDeclaration{private DelayQueue<RunDelayed> queue}
\fieldValue{new DelayQueue<RunDelayed>()}
\fieldComment{
	Prioritized FIFO queue for containing the RunDelayed items that have not expired. If execOnlyLast is true, then this queue should never contain more than one item.
}
\closeField

\beginField{workerThread}
\fieldDeclaration{private Thread workerThread}
\fieldValue{null}
\fieldComment{
	The worker thread that will run the inserted runnables. If the thread has no more work to do, it will set workerThread to null and terminate itself.
}
\closeField

\beginMethod{RunQueue()}
\methodDeclaration{public RunQueue()}
\methodComment{
	Creates an empty RunQueue with a delay of 0 and execOnlyLast set to false.
}
\closeMethod

\beginMethod{RunQueue(int)}
\methodDeclaration{public RunQueue(int delayMillis)}
\methodComment{
	Creates an empty RunQueue with execOnlyLast set to false.
}
\methodParam{delayMillis}{the length of execution delay in milliseconds; if less than 0, then 0 will be used.}
\closeMethod

\beginMethod{RunQueue(boolean)}
\methodDeclaration{public RunQueue(boolean execOnlyLast)}
\methodComment{
	Creates an empty RunQueue with a delay of 0.
}
\methodParam{execOnlyLast}{if true, only the last event will be executed after the delay; otherwise all are executed in order of appearance.}
\closeMethod

\beginMethod{RunQueue(int,boolean)}
\methodDeclaration{public RunQueue(int delayMillis, boolean execOnlyLast)}
\methodComment{
	Creates an empty RunQueue.
}
\methodParam{delayMillis}{the length of execution delay in milliseconds; if less than 0, then 0 will be used.}
\methodParam{execOnlyLast}{if true, only the last event will be executed after the delay; otherwise all are executed in order of appearance.}
\closeMethod

\beginMethod{isExecOnlyLast()}
\methodDeclaration{public synchronized boolean isExecOnlyLast()}
\methodReturn{true if only the last event will be executed after the delay; otherwise false.}
\closeMethod

\beginMethod{setExecOnlyLast(boolean)}
\methodDeclaration{public synchronized void setExecOnlyLast(boolean execOnlyLast)}
\methodParam{execOnlyLast}{if true, only the last event will be executed after the delay; otherwise all are executed in order of appearance.}
\closeMethod

\beginMethod{getDelayMillis()}
\methodDeclaration{public synchronized int getDelayMillis()}
\methodReturn{the delay in milliseconds}
\closeMethod

\beginMethod{setDelayMillis(int)}
\methodDeclaration{public synchronized void setDelayMillis(int delayMillis)}
\methodParam{delayMillis}{delay in milliseconds; if less than 0, then the new value is ignored.}
\closeMethod

\beginMethod{offer(Runnable)}
\methodDeclaration{public synchronized boolean offer(Runnable runnable)}
\methodComment{
	Inserts a Runnable object to the end of the queue. It will remain there until it is executed or another object replaces it. If execOnlyLast is set to true, the queue will be cleared before inserting this runnable to it. If there is no worker thread running, a new one will be spawned.
}
\methodParam{runnable}{the Runnable to be run after a pre-defined delay}
\methodReturn{true}
\methodThrows{NullPointerException}{if runnable is null}
\closeMethod

\beginMethod{join()}
\methodDeclaration{public synchronized void join()}
\methodComment{
	Waits for the queue to become empty.
}
\methodThrows{InterruptedException}{if another thread has interrupted the current thread. The interrupted status of the current thread is cleared when this exception is thrown.}
\closeMethod


\beginClass{RunQueue.RunQueueThread}
\classComment{
	Keeps on checking the RunQueue.queue to see if there are Runnables to be executed. If there is one, execute it and proceed to the next one. If an uncaught Throwable is thrown during the execution, prints an error message and stack trace to stderr. If the queue is empty, this thread will set RunDelayed.workerThread to null and terminate itself.
}
\classPackage{ikayaki.util}
\classDeclaration{private class RunQueueThread}
\classExtends{Thread}
\classCreatedBy{RunQueue}
\closeClass

\beginMethod{run()}
\methodDeclaration{public void run()}
\closeMethod


\beginClass{RunQueue.RunDelayed}
\classComment{
	Wraps a Runnable object and sets the delay after which it should be executed by a worker thread.
}
\classPackage{ikayaki.util}
\classDeclaration{private class RunDelayed}
\classImplements{Delayed}
\classCreatedBy{RunQueue}
\closeClass

\beginField{expires}
\fieldDeclaration{private long expires}
\fieldComment{
	The point in time when this RunDelayed will expire.
}
\closeField

\beginField{runnable}
\fieldDeclaration{private Runnable runnable}
\fieldComment{
	Contained Runnable object to be run after this RunDelayed has expired.
}
\closeField

\beginMethod{RunDelayed(Runnable,int)}
\methodDeclaration{public RunDelayed(Runnable runnable, int delayMillis)}
\methodComment{
	Creates a new RunDelayed item that contains runnable.
}
\methodParam{runnable}{the Runnable to be contained}
\methodParam{delayMillis}{delay in milliseconds}
\closeMethod

\beginMethod{getDelay(TimeUnit)}
\methodDeclaration{public long getDelay(TimeUnit unit)}
\methodComment{
	Returns the remaining delay associated with this object, always in milliseconds.
}
\methodParam{unit}{ignored; always assumed TimeUnit.MILLISECONDS}
\methodReturn{the remaining delay; zero or negative values indicate that the delay has already elapsed}
\closeMethod

\beginMethod{getRunnable()}
\methodDeclaration{public Runnable getRunnable()}
\methodComment{
	Returns the contained Runnable.
}
\methodReturn{the Runnable given as constructor parameter}
\closeMethod

\beginMethod{compareTo(Delayed)}
\methodDeclaration{public int compareTo(Delayed delayed)}
\methodComment{
	Compares this object with the specified object for order.  Returns a negative integer, zero, or a positive integer as this object is less than, equal to, or greater than the specified object.
}
\methodParam{delayed}{the Delayed to be compared.}
\methodReturn{a negative integer, zero, or a positive integer as this delay is less than, equal to, or greater than the specified delay.}
\closeMethod

