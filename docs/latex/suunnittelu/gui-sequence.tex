\beginClass{MeasurementSequencePanel}
\classComment{Allows creating, editing and removing measurement
sequences. Shows measurement data. Right-click brings popup menu for hiding columns, and saving
sequence. Left-click selects a row. Multiple rows can be selected by
painting or ctrl-clicking. Allows dragging rows to different order. Has three
textfields for inserting new sequences, first field for start value,
second for step and third for stop value. Clicking Add sequence-button
appends sequence into table. Saved sequences can be loaded from dropdown menu.}
\classPackage{ikayaki.gui}
\classDeclaration{public class MeasurementSequencePanel}
\classExtends{ProjectComponent}
\classCreatedBy{MainViewPanel}
\classUses{MeasurementSequenceTableModel}
\classUses{MeasurementSequencePopupMenu}
\classEvent{On SequenceTable mouse right-click}{Create a MeasurementSequencePopupMenu.}
\classEvent{On addSequence mouseclick}{Add measurement sequence to end
of table and tell MeasurementSequenceTableModel to update itself.}
\classEvent{On sequenceSelector mouseclick}{Bring dropdown menu for
selecting premade sequence.}
\classEvent{On selecting sequence from dropdown menu}{Add measurement
sequence to table and tell MeasurementSequenceTableModel to update itself.}
\classEvent{On Project event}{Update contest of table to correspond
projects state.}
\classEvent{On Measurement event}{If measurement step is finished, add
measurement data to appropriate row and if row being measured was
selected select next row unless measurement sequence ended.}
\closeClass

\beginField{addSequence}
\fieldDeclaration{private JButton addSequence}
\closeField

\beginField{sequenceSelector}
\fieldDeclaration{private JComboBox sequenceSelector}
\closeField

\beginField{sequenceStart}
\fieldDeclaration{private JTextField sequenceStart}
\closeField

\beginField{sequenceStep}
\fieldDeclaration{private JTextField sequenceStep}
\closeField

\beginField{sequenceStop}
\fieldDeclaration{private JTextField sequenceStop}
\closeField

\beginField{sequenceTable}
\fieldDeclaration{private JTable sequenceTable}
\closeField

\beginField{tableModel}
\fieldDeclaration{private MeasurementSequenceTableModel tableModel}
\closeField

\beginMethod{MeasurementSequencePanel()}
\methodDeclaration{public MeasurementSequencePanel()}
\methodComment{Creates default MeasurementSequencePanel.}
\closeMethod

\beginMethod{MeasurementSequencePanel(Project)}
\methodDeclaration{public MeasurementSequencePanel(Project project)}
\methodComment{Creates MeasurementSequencePanel and calculates shown
data from project's measurement data. Higlights right row.}
\closeMethod

\beginMethod{addSequence()}
\methodDeclaration{private void addSequence()}
\methodComment{Adds sequence determined by textfields to end of table.}
\closeMethod

\beginMethod{setProject(Project)}
\methodDeclaration{private void setProject(Project project)}
\methodComment{Calls super.setProject(project), clears table and calculates shown
data from project's measurement data. Higlights right row.}
\closeMethod

\beginClass{MeasurementSequenceTableModel}
\classComment{Handles data in table.}
\classPackage{ikayaki.gui}
\classDeclaration{public class MeasurementSequenceTableModel}
\classExtends{AbstractTableModel}
\classCreatedBy{MeasurementSequencePanel}
\closeClass

\beginField{shownColumns}
\fieldDeclaration{private Vector<TableColumns> shownColumns}
\fieldComment{Currently shown columns. }
\closeField

\beginField{allColumns}
\fieldDeclaration{private Vector<TableColumns> allColumns}
\fieldComment{All possible columns.}
\closeField

\beginMethod{MeasurementSequenceTableModel()}
\methodDeclaration{public MeasurementSequenceTableModel()}
\methodComment{Creates SequenceTableModel}
\closeMethod

\beginMethod{showColumn(String)}
\methodDeclaration{public void showColumn(String name)}
\methodComment{Shows named column.}
\methodParam{name}{name of the column to be shown}
\closeMethod

\beginMethod{hideColumn(String)}
\methodDeclaration{public void hideColumn(String name)}
\methodComment{Hides named column.}
\methodParam{name}{name of the column to be hidden}
\closeMethod


\beginClass{MeasurementSequencePopupMenu}
\classComment{Allows selection if volume is shown in table and saving sequence.}
\classPackage{ikayaki.gui}
\classDeclaration{public class MeasurementSequencePopupMenu}
\classExtends{JPopupMenu}
\classCreatedBy{MeasurementSequencePanel}
\closeClass

\beginField{volume}
\fieldDeclaration{private JCheckBox volume}
\closeField

\beginField{nameLabel}
\fieldDeclaration{private JLabel nameLabel}
\closeField

\beginField{nameTextField}
\fieldDeclaration{private JCheckBox nameTextField}
\closeField

\beginMethod{MeasurementSequencePopupMenu()}
\methodDeclaration{public MeasurementSequencePopupMenu()}
\methodComment{Creates SequencePopupMenu.}
\closeMethod

\beginMethod{saveSequence()}
\methodDeclaration{private void saveSequence()}
\methodComment{Saves current sequence into dropdown menu.}
\closeMethod
