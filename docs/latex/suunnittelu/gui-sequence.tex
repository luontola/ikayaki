\beginClass{MeasurementSequencePanel}
\classComment{Allows creating, editing and removing measurement
sequences. Shows measurement data. Right-click brings popup menu for hiding columns, and saving
sequence. Left-click selects a row. Multiple rows can be selected by
ctrl-clicking or shift-clicking. Allows dragging rows to different order
if multiple rows are selected multiple rows are dragged. Has three
textfields for inserting new sequences, first field for start value,
second for step and third for stop value. Clicking Add sequence-button
appends sequence into table. Saved sequences can be loaded from dropdown menu.}
\classPackage{ikayaki.gui}
\classDeclaration{public class MeasurementSequencePanel}
\classExtends{ProjectComponent}
\classCreatedBy{MainViewPanel}
\classUses{MeasurementSequenceTableModel}
\classUses{MeasurementSequencePopupMenu}
\classEvent{On SequenceTable mouse right-click}{Create a MeasurementSequencePopupMenu.}
\classEvent{On addSequence mouseclick}{Add measurement sequence to
project class and tell MeasurementSequenceTableModel to update itself.}
\classEvent{On sequenceSelector mouseclick}{Bring dropdown menu for
selecting premade sequence.}
\classEvent{On selecting sequence from dropdown menu}{Add measurement
sequence to table and tell MeasurementSequenceTableModel to update itself.}
\classEvent{On Project event}{Update contest of table to correspond
projects state.}
\classEvent{On Measurement event}{If measurement step is finished, get
measurement data from project class and if row being measured was
selected select next row unless measurement sequence ended.}
\classEvent{On Drag event}{Change measurement sequences row order in project class and
tell MeasurementSequenceTableModel to update itself to correspond new row
order. Order of rows with measurement data cannot be changed.}
\closeClass

\beginToc
\tocMethod{MeasurementSequencePanel()}
\tocMethod{addSequence()}
\tocMethod{setProject(Project)}
\closeToc

\beginField{addSequence}
\fieldDeclaration{private JButton addSequence}
\closeField

\beginField{sequenceSelector}
\fieldDeclaration{private JComboBox sequenceSelector}
\closeField

\beginField{sequenceStart}
\fieldDeclaration{private JTextField sequenceStart}
\closeField

\beginField{sequenceStep}
\fieldDeclaration{private JTextField sequenceStep}
\closeField

\beginField{sequenceStop}
\fieldDeclaration{private JTextField sequenceStop}
\closeField

\beginField{sequenceTable}
\fieldDeclaration{private JTable sequenceTable}
\closeField

\beginField{tableModel}
\fieldDeclaration{private MeasurementSequenceTableModel tableModel}
\closeField

\beginMethod{MeasurementSequencePanel()}
\methodDeclaration{public MeasurementSequencePanel()}
\methodComment{Creates default MeasurementSequencePanel.}
\closeMethod

\beginMethod{addSequence()}
\methodDeclaration{private void addSequence()}
\methodComment{Adds sequence determined by textfields to end of table.}
\closeMethod

\beginMethod{setProject(Project)}
\methodDeclaration{private void setProject(Project project)}
\methodComment{Calls super.setProject(project), clears table and calculates shown
data from project's measurement data.}
\closeMethod

\beginClass{MeasurementSequenceTableModel}
\classComment{Handles data in table.}
\classPackage{ikayaki.gui}
\classDeclaration{public class MeasurementSequenceTableModel}
\classExtends{AbstractTableModel}
\classCreatedBy{MeasurementSequencePanel}
\closeClass

\beginToc
\tocMethod{MeasurementSequenceTableModel()}
\tocMethod{showColumn(String)}
\tocMethod{hideColumn(String)}
\closeToc

\beginField{shownColumns}
\fieldDeclaration{private boolean volume}
\fieldComment{Tells if volume is shown in table.}
\closeField

\beginMethod{MeasurementSequenceTableModel()}
\methodDeclaration{public MeasurementSequenceTableModel()}
\methodComment{Creates SequenceTableModel}
\closeMethod

\beginMethod{showColumn(int)}
\methodDeclaration{public void showColumn(int name)}
\methodComment{Shows named column.}
\methodParam{name}{name of the column to be shown. possible values VOLUME=0}
\closeMethod

\beginMethod{hideColumn(int)}
\methodDeclaration{public void hideColumn(int name)}
\methodComment{Hides named column.}
\methodParam{name}{name of the column to be hidden. possible values VOLUME=0}
\closeMethod


\beginClass{MeasurementSequencePopupMenu}
\classComment{Allows selection if volume is shown in table and saving
sequence. Pops up when measurement sequence table is right-clicked.
Allows saving selected sequence or whole sequence.}
\classPackage{ikayaki.gui}
\classDeclaration{public class MeasurementSequencePopupMenu}
\classExtends{JPopupMenu}
\classCreatedBy{MeasurementSequencePanel}
\closeClass

\beginToc
\tocMethod{MeasurementSequencePopupMenu()}
\tocMethod{saveSequence()}
\closeToc

\beginField{volume}
\fieldDeclaration{private JCheckBox volume}
\fieldComment{If checked volume is shown in measurement sequence table.}
\closeField

\beginField{nameLabel}
\fieldDeclaration{private JLabel nameLabel}
\closeField

\beginField{nameTextField}
\fieldDeclaration{private JCheckBox nameTextField}
\fieldComment{Name of the sequence to be saved.}
\closeField

\beginMethod{MeasurementSequencePopupMenu()}
\methodDeclaration{public MeasurementSequencePopupMenu()}
\methodComment{Creates SequencePopupMenu.}
\closeMethod

\beginMethod{saveFullSequence()}
\methodDeclaration{private void saveSequence()}
\methodComment{Saves whole sequence into dropdown menu.}
\closeMethod

\beginMethod{saveSelectedSequence()}
\methodDeclaration{private void saveSequence()}
\methodComment{Saves selected sequence into dropdown menu.}
\closeMethod

\beginMethod{remove Rows()}
\methodDeclaration{private void removeRows()}
\methodComment{Removes selected rows. Rows with measurement data cannot
be removed.}
\closeMethod
