
\beginClass{Project}
\classPackage{ikayaki}
\classDeclaration{public class Project}
\classCreatedBy{ProjectExplorerPanel}
\classUses{MeasurementSequence}
\classUses{MeasurementStep}
\classUses{MeasurementResult}
\classUses{MeasurementValue}
\classUses{Squid}
\classUses{RunQueue}
\classUses{ProjectEvent}
\classUses{MeasurementEvent}
\classComment{
	Represents a measurement project file. Project is responsible for managing and storing the data that is recieved from the magnetometer measurements. Any changes made to the project will be written to file regularly (autosave).
	
	Project is responsible for controlling the magnetometer through the SQUID API. Controlling the SQUID will be done in a private worker thread. Only one project at a time may access the SQUID. 
}
\classPatterns{Facade}
\classEvent{On property change}{Autosaving will be invoked and the project written to file after a short delay.}
\classEvent{On measurement started/ended/paused/aborted}{ProjectEvent will be fired to all project listeners.}
\classEvent{On measurement subphase started/completed}{MeasurementEvent will be fired to all measurement listeners.}
\classEvent{On declination/inclination/volume changed}{The updated transformation matrix will be applied to all measurements and a ProjectEvent will be fired to all project listeners.}
\closeClass


\beginClass{Project.Type}
\classPackage{ikayaki}
\classDeclaration{public enum Type}
\classComment{
	The type of the project. Options are CALIBRATION, AF, THELLIER and THERMAL.
}
\closeClass


\beginClass{Project.State}
\classPackage{ikayaki}
\classDeclaration{public enum State}
\classComment{
	The state of the project's measurements. Options are IDLE, MEASURING, PAUSED, ABORTED.
}
\closeClass


\beginClass{MeasurementSequence}
\classPackage{ikayaki}
\classDeclaration{public class MeasurementSequence}
\classCreatedBy{Project}
\classUses{MeasurementStep}
\classComment{
	A list of measurement steps. Steps can be added or removed from the sequence.
}
\closeClass


\beginClass{MeasurementStep}
\classPackage{ikayaki}
\classDeclaration{public class MeasurementStep}
\classCreatedBy{Project, MeasurementSequencePanel}
\classUses{Project}
\classUses{MeasurementResult}
\classComment{
	A single step in a measurement sequence. Each step can include multiple measurements for improved measurement precision. A step can have a different volume and mass than the related project, but by default the volume and mass of the project will be used. Only the project may change the state and results of a measurement step.
}
\closeClass


\beginClass{MeasurementStep.State}
\classPackage{ikayaki}
\classDeclaration{public enum State}
\classComment{
	The state of a measurement step. Options are READY, MEASURING, DONE_RECENTLY and DONE.
}
\closeClass


\beginClass{MeasurementResult}
\classPackage{ikayaki}
\classDeclaration{public class MeasurementResult}
\classCreatedBy{Magnetometer}
\classComment{
	A set of X, Y and Z values measured by the magnetometer. The raw XYZ values will be rotated in 3D space by using a transformation matrix. The project will set and update the transformation whenever its parameters are changed.
}
\closeClass


\beginClass{MeasurementResult.Type}
\classPackage{ikayaki}
\classDeclaration{public enum Type}
\classComment{
	The orientation of the sample when it was measured. Options are BG, DEG0, DEG90, DEG180 and DEG270.
}
\closeClass

\beginMethod{getName()}
\methodDeclaration{public String getName()}
\methodReturn{"BG", "0", "90", "180" or "270"}
\closeMethod

\beginMethod{rotate(Tuple3d,Tuple3d)}
\methodDeclaration{public Tuple3d rotate(Tuple3d from, Tuple3d to)}
\methodComment{
	Rotates the raw XYZ values from the orientation of this object to that of DEG0. Rotating a BG or DEG0 will just copy the values directly.
}
\methodParam{from}{Old values that need to be rotated}
\methodParam{to}{Where the new values will be saved}
\methodReturn{The same as parameter to, or a new object if to was null.}
\closeMethod

\beginMethod{rotate(Tuple3d)}
\methodDeclaration{public Tuple3d rotate(Tuple3d from)}
\methodComment{
	Rotates the raw XYZ values from the orientation of this object to that of DEG0. Rotating a BG or DEG0 will just copy the values directly.
}
\methodParam{from}{Old values that need to be rotated}
\methodReturn{A new object with the rotated values.}
\closeMethod


\beginClass{MeasurementValue}
\classPackage{ikayaki}
\classDeclaration{public class MeasurementValue<T>}
\classExtends{}
\classImplements{}
\classCreatedBy{}
\classUses{}
\classUses{}
\classSubclass{}
\classSubclass{}
\classComment{
}
\classPatterns{}
\classEvent{}{}
\classEvent{}{}
\closeClass


\beginClass{ProjectEvent}
\classPackage{ikayaki}
\classDeclaration{}
\classExtends{}
\classImplements{}
\classCreatedBy{}
\classUses{}
\classUses{}
\classSubclass{}
\classSubclass{}
\classComment{
}
\classPatterns{}
\classEvent{}{}
\classEvent{}{}
\closeClass


\beginClass{ProjectListener}
\classPackage{ikayaki}
\classDeclaration{}
\classExtends{}
\classImplements{}
\classCreatedBy{}
\classUses{}
\classUses{}
\classSubclass{}
\classSubclass{}
\classComment{
}
\classPatterns{}
\classEvent{}{}
\classEvent{}{}
\closeClass


\beginClass{MeasurementEvent}
\classPackage{ikayaki}
\classDeclaration{}
\classExtends{}
\classImplements{}
\classCreatedBy{}
\classUses{}
\classUses{}
\classSubclass{}
\classSubclass{}
\classComment{
}
\classPatterns{}
\classEvent{}{}
\classEvent{}{}
\closeClass


\beginClass{MeasurementListener}
\classPackage{ikayaki}
\classDeclaration{}
\classExtends{}
\classImplements{}
\classCreatedBy{}
\classUses{}
\classUses{}
\classSubclass{}
\classSubclass{}
\classComment{
}
\classPatterns{}
\classEvent{}{}
\classEvent{}{}
\closeClass


