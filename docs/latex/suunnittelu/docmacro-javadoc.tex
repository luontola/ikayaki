
% Macros for generating .java files with javadoc

\pagestyle{empty}

\newcommand{\getMemA}[0]{}
\newcommand{\getMemB}[0]{}
\newcommand{\getMemC}[0]{}

% MACROS FOR CLASS DOCUMENTATION
\newcommand{\getClass}[0]{}
\newcommand{\beginClass}[1] {
	\subsubsection{#1}
	
	\renewcommand{\getClass}[0]{#1}
	\label{class:#1}
	
	\begin{hyphenrules}{nohyphenation}

/*\\
 \symbol{42} #1.java\\
 \symbol{42}\\
 \symbol{42} Copyright (C) 2005 Project SQUID, http://www.cs.helsinki.fi/group/squid/\\
 \symbol{42}\\
 \symbol{42} This file is part of Ikayaki.\\
 \symbol{42}\\
 \symbol{42} Ikayaki is free software; you can redistribute it and/or modify\\
 \symbol{42} it under the terms of the GNU General Public License as published by\\
 \symbol{42} the Free Software Foundation; either version 2 of the License, or\\
 \symbol{42} (at your option) any later version.\\
 \symbol{42}\\
 \symbol{42} Ikayaki is distributed in the hope that it will be useful,\\
 \symbol{42} but WITHOUT ANY WARRANTY; without even the implied warranty of\\
 \symbol{42} MERCHANTABILITY or FITNESS FOR A PARTICULAR PURPOSE.  See the\\
 \symbol{42} GNU General Public License for more details.\\
 \symbol{42}\\
 \symbol{42} You should have received a copy of the GNU General Public License\\
 \symbol{42} along with Ikayaki; if not, write to the Free Software\\
 \symbol{42} Foundation, Inc., 59 Temple Place, Suite 330, Boston, MA  02111-1307  USA\\
 \symbol{42}/\\
}
\newcommand{\classComment}[1] {
	\renewcommand{\getMemA}[0]{#1}
}
\newcommand{\classPackage}[1] {
	\renewcommand{\getMemB}[0]{#1}
}
\newcommand{\classDeclaration}[1] {
	package \getMemB;\\
	\\
	/**\\
	\symbol{42} \getMemA\\
	\symbol{42} \\
	\symbol{42} @author\\
	\symbol{42}/\\
	#1
}
\newcommand{\classExtends}[1] {
	extends #1
}
\newcommand{\classImplements}[1] {
	implements #1
}
\newcommand{\classCreatedBy}[1] {}
\newcommand{\classUses}[1] {
	\stepcounter{classUsesCounter}
}
\newcommand{\classSubclass}[1] {
	\stepcounter{classSubclassCounter}
}
\newcommand{\classPatterns}[1] {}
\newcommand{\classEvent}[2] {
	\stepcounter{classEventCounter}
	\\
	/*\\
	Event \Alph{classEventCounter}: #1 - #2\\
	\symbol{42}/\\
}
\newcommand{\closeClass}[0] {
	\end{hyphenrules}
	
	\setcounter{classUsesCounter}{0}
	\setcounter{classSubclassCounter}{0}
	\setcounter{classEventCounter}{0}
	
	\renewcommand{\getMemA}[0]{}
	\renewcommand{\getMemB}[0]{}
	\renewcommand{\getMemC}[0]{}
	
	\renewcommand{\isFirstField}[0]{}
	\renewcommand{\isFirstMethod}[0]{}
}
\newcounter{classUsesCounter}
\newcounter{classSubclassCounter}
\newcounter{classEventCounter}
\newcommand{\isFirstField}[0]{}
\newcommand{\isFirstMethod}[0]{}


% MACROS FOR CLASS TABLE OF CONTENTS
\newcommand{\beginToc}[0] {}
\newcommand{\tocMethod}[1] {}
\newcommand{\closeToc}[0] {}


% MACROS FOR FIELD DOCUMENTATION
\newcommand{\getField}[0]{}
\newcommand{\beginField}[1] {
	\isFirstField
	\renewcommand{\isFirstField}[0]{}
	
	\renewcommand{\getField}[0]{#1}
	\label{field:\getClass.#1}
	
	\begin{hyphenrules}{nohyphenation}
	
}
\newcommand{\fieldDeclaration}[1] {
	\renewcommand{\getMemA}[0]{#1}
}
\newcommand{\fieldValue}[1] {
	\renewcommand{\getMemB}[0]{ = #1}
}
\newcommand{\fieldComment}[1] {
	/**\\
	\symbol{42} #1\\
	\symbol{42}/\\
}
\newcommand{\closeField}[0] {
	\getMemA\getMemB;\\
	
	\end{hyphenrules}
	
	\renewcommand{\getMemA}[0]{}
	\renewcommand{\getMemB}[0]{}
	\renewcommand{\getMemC}[0]{}
}


% MACROS FOR METHOD DOCUMENTATION
\newcommand{\getMethod}[0]{}
\newcommand{\beginMethod}[1] {
	\isFirstMethod
	\renewcommand{\isFirstMethod}[0]{}
	
	\renewcommand{\getMethod}[0]{#1}
	\label{method:\getClass.#1}
	
	\begin{hyphenrules}{nohyphenation}
	/**\\
}
\newcommand{\methodDeclaration}[1] {
	\renewcommand{\getMemA}[0]{#1}
}
\newcommand{\methodComment}[1] {
	\symbol{42} #1\\
}
\newcommand{\methodParam}[2] {
	\stepcounter{methodParamCounter}
	\symbol{42} @param #1 #2 \\
}
\newcommand{\methodReturn}[1] {
	\symbol{42} @return #1 \\
}
\newcommand{\methodThrows}[2] {
	\symbol{42} @throws #1 #2 \\
}
\newcommand{\closeMethod}[0] {
	\symbol{42}/\\
	\getMemA \{\\
	return null; // TODO\\
	\}\\
	
	\end{hyphenrules}
	\setcounter{methodParamCounter}{0}
	
	\renewcommand{\getMemA}[0]{}
	\renewcommand{\getMemB}[0]{}
	\renewcommand{\getMemC}[0]{}
}
\newcounter{methodParamCounter}

