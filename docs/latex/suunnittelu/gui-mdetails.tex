\beginClass{MeasurementDetailsPanel}
\classComment{Shows details of measurement selected in
MeasurementSequencePanel.}
\classPackage{ikayaki.gui}
\classDeclaration{public class MeasurementDetailsPanel}
\classExtends{ProjectComponent}
\classCreatedBy{MainViewPanel}
\classEvent{On project event}{Update tables to correspond projects new state.}
\classEvent{On change of selected row in MeasurementSequencePanel}{Change
tables to correspond selected row.}
\classEvent{On measurement event}{If row corresponding to ongoing measurement
is selected in MeasurementSequencePanel update tables with new
measurement data.}
\closeClass

\beginToc
\closeToc

\beginField{measurementDetails}
\fieldDeclaration{private JTable measurementDetails}
\fieldComment{X, Y and Z components of BG1, 0, 90, 180, 270, BG2}
\closeField

\beginField{errorDetails}
\fieldDeclaration{private JTable errorDetails}
\fieldComment{S/D, S/H and S/N of error}
\closeField

\beginField{tableModel}
\fieldDeclaration{private DefaultTableModel tableMoled}
\closeField

\beginMethod{MeasurementDetails()}
\methodDeclaration{public MeasurementDetails()}
\methodComment{Creates default MeasurementDetailsPanel.}
\closeMethod

\beginMethod{MeasurementDetails(Project)}
\methodDeclaration{public MeasurementDetails(Project project)}
\methodComment{Creates MeasurementDetailsPanel with measurement details taken
from project.}
\closeMethod

\beginMethod{setProject(Project)}
\methodDeclaration{private void setProject(Project project)}
\methodComment{Calls super.setProject(project), clears tables and
shows new projects measurement details.}
\closeMethod
