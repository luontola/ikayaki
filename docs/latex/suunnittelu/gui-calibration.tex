\beginClass{CalibrationPanel}
\classComment{
	Holds predefined "Holder noise" and "Standard sample" projects for calibration; they are in a technically same table as Project explorer files. Also has a "Calibrate" button, which executes selected calibration project, similarly to clicking "Single step" in normal projects.
}
\classPackage{ikayaki.gui}
\classDeclaration{public class CalibrationPanel}
\classExtends{ProjectComponent}
\classCreatedBy{MainViewPanel}
\classUses{MainViewPanel}
\classUses{Project}
\classEvent{On calibrateButton click}{call project.doSingleStep(); show error message if false is returned.}
\classEvent{On calibrationProjectTable click}{call Project.loadProject(File) with clicked project file (calibrationProjectTable row); call MainViewPanel.changeProject(Project) with returned Project unless null, on which case show error message and revert calibrationProjectTable selection to old project, if any.}
\classEvent{On ProjectEvent}{hilight calibration project whose measuring started, or unhilight one whose measuring ended; enable calibrateButton if measuring has ended, or disable if measuring has started.}
\closeClass

\beginField{calibrateButton}
\fieldDeclaration{private JButton calibrateButton}
\closeField

\beginField{calibrationProjectTable}
\fieldDeclaration{private JTable calibrationProjectTable}
\fieldComment{Table for the two calibration projects; has "filename", "last modified" and "time" (time since last modification) columns.}
\closeField

\beginField{calibrationProjectTableModel}
\fieldDeclaration{private TableModel calibrationProjectTableModel}
\fieldComment{TableModel which holds the data for calibration projects. Unnamed inner class.}
\closeField

\beginMethod{setProject(Project)}
\methodDeclaration{public void setProject(Project project)}
\methodComment{Call super.setProject(project), hilight selected calibration project, or unhilight unselected calibration project.}
\closeMethod
