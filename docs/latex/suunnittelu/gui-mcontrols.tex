\beginClass{MeasurementControlsPanel}
\classComment{
	Has "Measure"/"Pause", "Single step" and "Stop now!" buttons for controlling measurements; "+z/-z" radiobuttons for changing sample orientation, help picture for inserting sample, picture of current magnetometer status, and, manual controls.
	Listens MeasurementEvents and ProjectEvents, and updates buttons and magnetometer status accordingly.
}
\classPackage{ikayaki.gui}
\classDeclaration{public class MeasurementControlsPanel}
\classExtends{ProjectComponent}
\classCreatedBy{MainViewPanel}
\classUses{Project}
\classUses{MagnetometerStatusPanel}
\classUses{ManualControlsPanel}
\classEvent{On measureButton click}{call project.doAutoStep() or project.doPause(), depending on current button status. Show error message if false is returned.}
\classEvent{On singlestepButton click}{call project.doSingleStep(); show error message if false is returned.}
\classEvent{On stopButton click}{call project.doAbort(); show critical error message if false is returned.}
\classEvent{On ProjectEvent}{update buttons and manual controls according to project.isXXXEnabled().}
\classEvent{On MeasurementEvent}{call magnetometerStatusPanel.updateStatus(int, int) with the right values from MeasurementEvent.}
\closeClass

\beginField{measureButton}
\fieldDeclaration{private JButton measureButton}
\fieldComment{Measure/pause -button; "Measure" when no measuring is being done, "Pause" when there is ongoing measuring sequence.}
\closeField

\beginField{singlestepButton}
\fieldDeclaration{private JButton singlestepButton}
\closeField

\beginField{stopButton}
\fieldDeclaration{private JButton stopButton}
\closeField

\beginField{zButtonGroup}
\fieldDeclaration{private ButtonGroup zButtonGroup}
\fieldComment{Groups together +z and -z RadioButtons.}
\closeField

\beginField{zPlusRadioButton}
\fieldDeclaration{private JRadioButton zPlusRadioButton}
\closeField

\beginField{zMinusRadioButton}
\fieldDeclaration{private JRadioButton zMinusRadioButton}
\closeField

\beginField{sampleInsertPanel}
\fieldDeclaration{private JPanel sampleInsertPanel}
\fieldComment{Draws a help image for sample inserting.}
\closeField

\beginField{magnetometerStatusPanel}
\fieldDeclaration{private MagnetometerStatusPanel magnetometerStatusPanel}
\closeField

\beginField{manualControlsPanel}
\fieldDeclaration{private ManualControlsPanel manualControlsPanel}
\closeField


\beginClass{MagnetometerStatusPanel}
\classComment{Picture of current magnetometer status, with sample holder position and rotation. Status is updated according to MeasurementEvents received by MeasurementControlsPanel.}
\classPackage{ikayaki.gui}
\classDeclaration{public class MagnetometerStatusPanel}
\classExtends{JPanel}
\classCreatedBy{MeasurementControlsPanel}
\closeClass

\beginMethod{MagnetometerStatusPanel()}
\methodDeclaration{public MagnetometerStatusPanel()}
\methodComment{Sets magnetometer status to current position.}
\closeMethod

\beginMethod{updateStatus(int,int)}
\methodDeclaration{public void updateStatus(int position, int rotation)}
\methodComment{Updates magnetometer status picture; called by MeasurementControlsPanel when it receives MeasurementEvent.}
\methodParam{position}{sample holder position, from 1 to 16777215.}
\methodParam{rotation}{sample holder rotation, from 0 (angle 0) to 2000 (angle 360).}
\closeMethod


\beginClass{ManualControlsPanel}
\classComment{Magnetometer manual controls. MeasurementControlsPanel disables these whenever a normal measurement step is going.}
\classPackage{ikayaki.gui}
\classDeclaration{public class ManualControlsPanel}
\classExtends{JPanel}
\classCreatedBy{MeasurementControlsPanel}
\classUses{Project}
\classEvent{On xxxN click}{call project.xxxN()...}
\closeClass

\beginField{moveButtonGroup}
\fieldDeclaration{private ButtonGroup moveButtonGroup}
\fieldComment{Groups together all sample holder moving RadioButtons (moveXXX).}
\closeField

\beginField{moveHome}
\fieldDeclaration{private JRadioButton moveHome}
\fieldComment{Moves sample holder to home position.}
\closeField

\beginField{moveDemagZ}
\fieldDeclaration{private JRadioButton moveDemagZ}
\fieldComment{Moves sample holder to demagnetize-Z position.}
\closeField

\beginField{moveDemagY}
\fieldDeclaration{private JRadioButton moveDemagY}
\fieldComment{Moves sample holder to demagnetize-Y position.}
\closeField

\beginField{moveBG}
\fieldDeclaration{private JRadioButton moveBG}
\fieldComment{Moves sample holder to background position.}
\closeField

\beginField{moveMeasure}
\fieldDeclaration{private JRadioButton moveMeasure}
\fieldComment{Moves sample holder to measurement position.}
\closeField

\beginField{rotateButtonGroup}
\fieldDeclaration{private ButtonGroup rotateButtonGroup}
\fieldComment{Groups together all sample holder rotating RadioButtons (rotateXXX).}
\closeField

\beginField{rotate0}
\fieldDeclaration{private JRadioButton rotate0}
\fieldComment{Rotates sample holder to angle 0.}
\closeField

\beginField{rotate90}
\fieldDeclaration{private JRadioButton rotate90}
\fieldComment{Rotates sample holder to angle 90.}
\closeField

\beginField{rotate180}
\fieldDeclaration{private JRadioButton rotate180}
\fieldComment{Rotates sample holder to angle 180.}
\closeField

\beginField{rotate270}
\fieldDeclaration{private JRadioButton rotate270}
\fieldComment{Rotates sample holder to angle 270.}
\closeField

\beginField{measureAllButton}
\fieldDeclaration{private JButton measureAllButton}
\fieldComment{Measures X, Y and Z (at current sample holder position) by calling project.doMeasureXYZ().}
\closeField

\beginField{resetAllButton}
\fieldDeclaration{private JButton resetAllButton}
\fieldComment{Resets X, Y and Z? Does what?}
\closeField

\beginField{demagZButton}
\fieldDeclaration{private JButton demagZButton}
\fieldComment{Demagnetizes in Z (at current sample holder position) by calling project.doDemagZ().}
\closeField

\beginField{demagYButton}
\fieldDeclaration{private JButton demagYButton}
\fieldComment{Demagnetizes in Y (at current sample holder position) by calling project.doDemagY().}
\closeField
