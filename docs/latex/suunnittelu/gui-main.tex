\beginClass{Ikayaki}
\classComment{
	Starts the program. Lays out MainViewPanel, MainMenuBar and MainStatusBar in a JFrame.
}
\classPackage{ikayaki.gui}
\classDeclaration{public class Ikayaki}
\classExtends{JFrame}
\classUses{MainViewPanel}
\classUses{MainMenuBar}
\classUses{MainStatusBar}
\classEvent{On window close}{checks that no measurement is running. Saves all opened project files and settings. Closes the program, or notifies the user if the program may not be closed.}
\closeClass

\beginToc
\tocMethod{main(String[])}
\closeToc

\beginMethod{main(String[])}
\methodDeclaration{public static void main(String[] args)}
\methodComment{
	Starts the program with the provided command line parameters. If the location of a project file is given as a parameter, the program will try to load it.
}
\methodParam{args}{command line parameters.}
\closeMethod


\beginClass{MainViewPanel}
\classComment{
    Creates the main view panels (split panels) and Squid and Project components. It also tells everybody if the current project is changed. 
}
\classPackage{ikayaki.gui}
\classDeclaration{public class MainViewPanel}
\classExtends{JPanel}
\classCreatedBy{Ikayaki}
\classUses{ProjectExplorerPanel}
\classUses{CalibrationPanel}
\classUses{Squid}
\classUses{MainMenuBar}
\classUses{MainStatusBar}
\classUses{ProjectInformationPanel}
\classUses{MeasurementSequencePanel}
\classUses{MeasurementDetailsPanel}
\classUses{MeasurementControlsPanel}
\classUses{MeasurementGraphsPanel}
\closeClass

\beginToc
\tocMethod{MainViewPanel()}
\tocMethod{changeProject(Project)}
\closeToc

\beginField{projectExplorer}
\fieldDeclaration{private ProjectExplorerPanel projectExplorer}
\closeField

\beginField{calibration}
\fieldDeclaration{private CalibrationPanel calibration}
\closeField

\beginField{squid}
\fieldDeclaration{private Squid squid}
\closeField

\beginField{project}
\fieldDeclaration{private Project project}
\fieldComment{Currently opened project.}
\closeField

\beginField{measuringProject}
\fieldDeclaration{private Project measuringProject}
\fieldComment{Project which has an ongoing measurement, or null if no measurement is running.}
\closeField

\beginField{menuBar}
\fieldDeclaration{private MainMenuBar menuBar}
\closeField

\beginField{statusBar}
\fieldDeclaration{private MainStatusBar statusBar}
\closeField

\beginField{projectInformation}
\fieldDeclaration{private ProjectInformationPanel projectInformation}
\closeField

\beginField{measurementSequence}
\fieldDeclaration{private MeasurementSequencePanel measurementSequence}
\closeField

\beginField{measurementControls}
\fieldDeclaration{private MeasurementControlsPanel measurementControls}
\closeField

\beginField{measurementDetails}
\fieldDeclaration{private MeasurementDetailsPanel measurementDetails}
\closeField

\beginField{measurementGraphs}
\fieldDeclaration{private MeasurementGraphsPanel measurementGraphs}
\closeField

\beginMethod{MainViewPanel()}
\methodDeclaration{public MainViewPanel()}
\methodComment{
    Loads default view and creates all components and panels. Splitpanel between Calibration,Explorer,Information and rest.
}
\closeMethod

\beginMethod{changeProject(Project)}
\methodDeclaration{public boolean changeProject(Project project)}
\methodComment{
    Looks for file with filename, if not exist creates new other wise opens it. Then updates current project and tells Panels new project is opened.
}
\closeMethod
