\beginClass{Settings}
\classPackage{ikayaki}
\classDeclaration{public class Settings}
\classExtends{}
\classCreatedBy{Ikayaki}
\classComment{This singleton class has all necessary settings for Squid and program to run. It also holds sequences.}
\classPatterns{This class is singleton, there will be always only one instance of it and its created when class is needed first time.}

\closeClass

\beginField{properties}
\fieldDeclaration{private Properties properties}
\fieldComment{All properties in Map: magnetometerPort,demagnetizer,PorthandlerPort,xAxisCalibration,yAxisCalibration,zAxisCalibration,demagRamp,demagDelay,acceleration,deceleration,velocity,measurementVelocity,transverseYPosition,axialPosition,sampleLoadPosition,backgroundPosition,measurementPosition,rotation,handlerRightLimit}
\closeField

\beginField{propertiesFile}
\fieldDeclaration{private File propertiesFile}
\fieldComment{filename for properties (xml)}
\closeField

\beginField{sequences}
\fieldDeclaration{private List sequences}
\fieldComment{all saved sequences}
\closeField

\beginField{sequencesModified}
\fieldDeclaration{private bool sequencesModified}
\fieldComment{we save them only if they have been modified}
\closeField

\beginField{sequencesFile}
\fieldDeclaration{private File sequencesFile}
\fieldComment{filename for properties (xml)}
\closeField

\beginField{autoSaveQueue}
\fieldDeclaration{private RunQueue autoSaveQueue}
\fieldComment{autosave once a while}
\closeField

\beginMethod{Settings()}
\methodDeclaration{private Settings()}
\methodComment{
    loads settings from config-file.
}
\closeMethod

\beginMethod{instance()}
\methodDeclaration{public Settings instance()}
\methodComment{
    creates Settings, if not yet created, and return it.
}
\closeMethod

\beginMethod{getProperty(String)}
\methodDeclaration{public String getProperty(String key)}
\methodComment{
    returns value with key, or null if not exists.
}
\closeMethod

\beginMethod{setProperty(String,String)}
\methodDeclaration{public void setProperty(String key, String value)}
\methodComment{
    adds pair to properties.
}
\closeMethod

\beginMethod{getSequences()}
\methodDeclaration{public MeasurementSequence[] getSequences()}
\methodComment{
    returns all Sequences.
}
\closeMethod

\beginMethod{addSequence(MeasurementSequence sequence)}
\methodDeclaration{public bool addSequence(MeasurementSequence sequence)}
\methodComment{
    add sequence to sequence list, return true on success.
}
\closeMethod

\beginMethod{removeSequence(MeasurementSequence sequence)}
\methodDeclaration{public void removeSequence(MeasurementSequence sequence)}
\methodComment{
    add sequence to sequence list, return true on success.
}
\closeMethod

\beginMethod{save()}
\methodDeclaration{public void save()}
\methodComment{
    saves after a while when no changes have come.
}
\closeMethod

\beginMethod{saveNow()}
\methodDeclaration{public void saveNow()}
\methodComment{
    saves and keeps waiting until its done.
}
\closeMethod

