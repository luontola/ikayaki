\beginClass{Settings}
\classComment{
    Singleton class for holding all global settings. It also holds saved sequences. All changes are automatically saved to file after a short delay. 
}
\classPackage{ikayaki}
\classDeclaration{public class Settings}
\classPatterns{Singleton}
\closeClass

\beginField{properties}
\fieldDeclaration{private Properties properties}
\fieldValue{new Properties()}
\fieldComment{All properties in a map. Keys are: magnetometerPort(String), demagnetizerPort(String), PorthandlerPort(String), xAxisCalibration(double), yAxisCalibration(double), zAxisCalibration(double), demagRamp(int), demagDelay(int), acceleration(int), deceleration(int), velocity(int), measurementVelocity(int), transverseYPosition(int), axialPosition(int), sampleLoadPosition(int), backgroundPosition(int), measurementPosition(int), rotation(int), handlerRightLimit(boolean)}
\closeField

\beginField{propertiesFile}
\fieldDeclaration{private File propertiesFile}
\fieldComment{File where the properties will be saved in XML format}
\closeField

\beginField{propertiesModified}
\fieldDeclaration{private boolean propertiesModified}
\fieldComment{true if the properties have been modified, otherwise false}
\closeField

\beginField{sequences}
\fieldDeclaration{private List<MeasurementSequence> sequences}
\fieldValue{new ArrayList<MeasurementSequence>()}
\fieldComment{All saved sequences}
\closeField

\beginField{sequencesFile}
\fieldDeclaration{private File sequencesFile}
\fieldComment{File where the sequences will be saved in XML format}
\closeField

\beginField{sequencesModified}
\fieldDeclaration{private bool sequencesModified}
\fieldComment{true if the sequences have been modified, otherwise false}
\closeField

\beginField{autoSaveQueue}
\fieldDeclaration{private RunQueue autoSaveQueue}
\fieldComment{Queue for scheduling save operations after properties/sequences have been changed}
\closeField

\beginMethod{instance()}
\methodDeclaration{public static Settings instance()}
\methodReturn{Pointer to the global Settings object. If not yet created, will first create a new Settings object.}
\closeMethod

\beginMethod{Settings()}
\methodDeclaration{private Settings()}
\methodComment{
    Loads settings from configuration files.
}
\closeMethod

\beginMethod{save()}
\methodDeclaration{public void save()}
\methodComment{
    Saves the settings after a while when no changes have come.
}
\closeMethod

\beginMethod{saveNow()}
\methodDeclaration{public void saveNow()}
\methodComment{
    Saves the settings and keeps waiting until its done.
}
\closeMethod

\beginMethod{getProperty(String)}
\methodDeclaration{private String getProperty(String key)}
\methodComment{
    Returns the value that maps to the specified key.
}
\methodParam{key}{key whose associated value is to be returned.}
\methodReturn{Value associated with key, or an empty string if none exists.}
\closeMethod

\beginMethod{setProperty(String,String)}
\methodDeclaration{private void setProperty(String key, String value)}
\methodComment{
    Associates the specified value with the specified key. Will invoke autosaving.
}
\methodParam{key}{key with which the specified value is to be associated.}
\methodParam{value}{value to be associated with the specified key.}
\closeMethod

\beginMethod{getXXX()}
\methodDeclaration{public Type getXXX()}
\methodComment{
    Generic accessor for all properties. Returns the value from Properties in appropriate type.
}
\methodReturn{Value associated with key}
\closeMethod

\beginMethod{setXXX(Type value)}
\methodDeclaration{public boolean setXXX(Type value)}
\methodComment{
    Generic accessor for all properties. Invoke autosave. Checks if value is ok and sets it.
}
\methodReturn{True if value was ok.}
\closeMethod

\beginMethod{getSequences()}
\methodDeclaration{public MeasurementSequence[] getSequences()}
\methodComment{
    Returns all saved Sequences.
}
\closeMethod

\beginMethod{addSequence(MeasurementSequence sequence)}
\methodDeclaration{public void addSequence(MeasurementSequence sequence)}
\methodComment{
    Adds a sequence to the sequence list.
}
\closeMethod

\beginMethod{removeSequence(MeasurementSequence sequence)}
\methodDeclaration{public void removeSequence(MeasurementSequence sequence)}
\methodComment{
    Removes a sequence from the sequence list. If the specified sequence is not in the list, it will be ignored.
}
\closeMethod

