\beginClass{SerialIO}
\classComment{This class represents hardware layer to serial port communications.}
\classPackage{ikayaki.squid}
\classDeclaration{public class SerialIO}
\classImplements{SerialPortEventListener}
\classCreatedBy{Squid}
\classEvent{On new SerialPortEvent}{generates new SerialMessageArrivedEvent if a data message from serial port is received.}
\closeClass

\beginField{lastMessage}
\fieldDeclaration{private String lastMessge}
\fieldValue{null}
\fieldComment{contains last received message from the serial port that this SerialIO represents.}
\closeField

\beginMethod{SerialIO}
\methodDeclaration{public void SerialIO(SerialParameters parameters)}
\methodComment{Creates an instance of SerialIO which represents one serial port.}
\methodParam{parameters}{parameters for the serial port being opened.}
\methodReturn{SerialIO object.}
\methodThrows{NoSuchPortException}{if no such port is found.}
\methodThrows{PortInUseException}{if the serial port is already in use.}
\closeMethod

\beginMethod{writeMessage}
\methodDeclaration{public writeMessage(String message)}
\methodComment{Writes an ASCII format message to serial port.}
\methodParam{message}{message to be send}
\methodThrows{NoSuchPortException}{if no such port is found.}
\methodThrows{PortInUseException}{if serial port is already in use.}
\closeMethod

\beginMethod{getLastAnswer}
\methodDeclaration{public String getLastAnswer()}
\methodComment{Writes an ASCII format message to serial port. SerialIO sends andSerialPortEvent if it gets answer to this message.}
\methodReturn{last answer receiver from serial port or null if no last message is available.}
\closeMethod


\beginClass{SerialParameters}
\classComment{Contains all the serial communication parameters which SerialIO uses when opening the port.}
\classPackage{ikayaki.squid}
\classDeclaration{public class SerialParameters}
\classCreatedBy{Squid}
\closeClass

\beginField{portName}
\fieldDeclaration{private String}
\fieldComment{The name of the serial port.}
\closeField

\beginField{baudRate}
\fieldDeclaration{private int}
\fieldComment{The baud rate.}
\closeField

\beginField{flowControlIn}
\fieldDeclaration{private int}
\fieldComment{Type of flow control for receiving.}
\closeField

\beginField{flowControlOut}
\fieldDeclaration{private int}
\fieldComment{Type of flow control for sending.}
\closeField

\beginField{databits}
\fieldDeclaration{private int}
\fieldComment{The number of data bits.}
\closeField

\beginField{stopbits}
\fieldDeclaration{private int}
\fieldComment{The number of stop bits.}
\closeField

\beginField{parity}
\fieldDeclaration{private int}
\fieldComment{The type of parity.}
\closeField

\beginMethod{SerialParameters}
\methodDeclaration{public SerialParameters(String portName, int baudRate, int flowControlIn, int flowControlOut, int databits, int stopbits, int parity)}
\methodComment{Creates a SerialParameter object containing settings for serial port communication.}
\methodReturn{Parameter settings object.}
\methodParam{portName}{The name of the serial port.}
\methodParam{baudRate}{The baud rate.}
\methodParam{flowControlIn}{Type of flow control for receiving.}
\methodParam{flowControlOut}{Type of flow control for sending.}
\methodParam{databits}{The number of data bits.}
\methodParam{stopbits}{The number of stop bits.}
\methodParam{parity}{The type of parity.}
\closeMethod


\beginClass{SerialMessageEvent}
\classComment{An event that is generated when SerialIO receives data from serial port. }
\classPackage{ikayaki.squid}
\classDeclaration{public class SerialMessageEvent extends java.util.EventObject}
\classCreatedBy{SerialIO}
\closeClass

\beginField{message}
\fieldDeclaration{private String}
\fieldComment{ASCII message recieved from serial port.}
\closeField

\beginMethod{getMessage}
\methodDeclaration{public String getMessage()}
\methodComment{Returns received serial message.}
\methodReturn{The message in ASCII form that was received from serial port.}
\closeMethod

\beginClass{SerialMessageListener}
\classComment{If a class wants to receive SerialMessageEvents it must implement this interface.}
\classPackage{ikayaki.squid}
\classDeclaration{public abstract interface SerialMessageListener extends java.util.EventListener}
\classCreatedBy{SerialIO}
\closeClass

\beginMethod{SerialMessageEvent}
\methodDeclaration{void serialEvent(SerialMessageEvent event) }
\methodComment{Propagates serial port message event.}
\methodParam{event}{the event that happened.}
\closeMethod