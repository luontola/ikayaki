\beginClass{SquidEmulator}
\classComment{
    This class tries to emulate behavior of real squid-system. It starts 3 threads (handler,magnetometer,degausser), opens COM-ports for 
them and adds SerialIO Listeners.  Threads generates random data values or loaded values as results and generates random error 
situations 
to see that program 
using real squid system does survive those. Uses 2-3 COM ports. Usage SquidEmulator x z.. filename where x is 0 or 1 and indicates if 
Magnetometer and Demagnetizer are on same COM port. z... values are COM ports. filename is name of log file we are using or it is 
existing log file, which is used to generate same sequence used to verify that old and new program behaves same way.
}
\classPackage{ikayaki.squid}
\classDeclaration{public class SquidEmulation}
\classUses{SerialIO}
\classEvent{On New IO Message}{reads message and puts it in Buffer}
\closeClass

\beginToc
\tocMethod{writeMessage(String,int)}
\tocMethod{run()}
\closeToc

\beginField{online}
\fieldDeclaration{private bool online}
\fieldComment{indicates if system have been started}
\closeField

\beginField{logFile}
\fieldDeclaration{private File logFile}
\fieldComment{log file we are using read or write}
\closeField

\beginField{usingOldLog}
\fieldDeclaration{private boolean usingOldLog}
\fieldComment{indicates have we loaded log file for using or are we writing it}
\closeField

\beginField{acceleration}
\fieldDeclaration{private int acceleration}
\fieldComment{value between 0 and 127 default 5. Settings in the
20-50 range are usually employed.}
\closeField

\beginField{deceleration}
\fieldDeclaration{private int deceleration}
\fieldComment{value between 0 and 127 default 10. Settings in the
20-50 range are usually employed.}
\closeField

\beginField{velocity}
\fieldDeclaration{private int velocity}
\fieldComment{value between 50 and 12 000. The
decimal number issued is 10 times the actual pulse rate to the motor. Since the
motor requires 200 pulses (full step) or 400 pulses (half step) per revolution,
a speed setting of M10000 sets the motor to revolve at 5 revolutions per second
in full step or 2.5 revolutions in half step. This rate is one-half the sample
rate rotation due to the pulley ratios. The sample handler is set up at the
factory for half stepping.}
\closeField

\beginField{handlerStatus}
\fieldDeclaration{private String handlerStatus}
\fieldComment{5 end of move, previous G command complete, 7 hard limit stop, G motor is currently indexing}
\closeField

\beginField{commandedDistance}
\fieldDeclaration{private int commandedDistance}
\fieldComment{value between 1 and 16,777,215}
\closeField

\beginField{currentPosition}
\fieldDeclaration{private int currentPosition}
\fieldComment{value between 1 and 16,777,215}
\closeField

\beginField{homePosition}
\fieldDeclaration{private int homePosition}
\fieldComment{value between 1 and 16,777,215}
\closeField

\beginField{commandedRotation}
\fieldDeclaration{private int commandedRotation}
\fieldComment{angles are between 0 (0) and 2000 (360)}
\closeField

\beginField{currentRotation}
\fieldDeclaration{private int currentRotation}
\fieldComment{angles are between 0 (0) and 2000 (360)}
\closeField

\beginField{degausserCoil}
\fieldDeclaration{private int degausserCoil}
\fieldComment{(X, Y, Z) = (0,1,2) default axis Z}
\closeField

\beginField{degausserAmplitude}
\fieldDeclaration{private int degausserAmplitude}
\fieldComment{0->3000 default amp 0}
\closeField

\beginField{degausserDelay}
\fieldDeclaration{private int degausserDelay}
\fieldComment{1-9 seconds default delay 1 second}
\closeField

\beginField{degausserRamp}
\fieldDeclaration{private int degausserRamp}
\fieldComment{(3, 5, 7, 9) default 3}
\closeField

\beginField{degausserStatus}
\fieldDeclaration{private char degausserRamp}
\fieldComment{Z=Zero, T=Tracking, ?=Unknown}
\closeField

\beginField{serialIO}
\fieldDeclaration{private SerialIO[] serialIO} 
\fieldComment{starts
Threads which reads messages from selected COM port. Own listener
for each. Offers write commads to port too.}
\closeField

\beginField{handler}
\fieldDeclaration{private HandlerEmu handler}
\fieldComment{private class which implements Thread and runs handler emulation process. Process incoming messages and sends data back. When message comes, process it (wait if needed for a while), updates own status and sends result back.}
\closeField

\beginField{magnetometer}
\fieldDeclaration{private MagnetometerEmu magnetometer}
\fieldComment{private class which implements Thread and runs
magnetometer emulation process. Process incoming messages and sends
data back. When message comes, process it (wait if needed for a while), updates own status and sends result back.}
\closeField

\beginField{degausser}
\fieldDeclaration{private DegausserEmu degausser}
\fieldComment{private class which implements Thread and runs
degausser emulation process. Process incoming messages and sends
data back. When message comes, process it (wait if needed for a while), updates own status and sends result back.}
\closeField

\beginMethod{writeMessage(String,int)}
\methodDeclaration{public void writeMessage(String message ,int port))}
\methodComment{
    send message to SerialIO to be sented.
}
\methodParam{message}{any message reply we are sending back}
\methodParam{port}{port number to be sent}
\closeMethod

\beginMethod{main(String args)}
\methodDeclaration{public void main(String args))}
\methodComment{
    First creates or loads log file and sets settings .Runs sequence where read data from buffer and run cheduled actions (move, rotate, 
demag, measure) and send feedback to COM ports.
}
\closeMethod
