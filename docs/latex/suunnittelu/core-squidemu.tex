\beginClass{SquidEmulator}
\classPackage{ikayaki.squid}
\classDeclaration{public class SquidEmulation}
\classExtends{Thread}
\classCreatedBy{}
\classUses{SerialIO}
\classComment{
	This class tries to emulate behavior of real squid-system. It generates random data values as results and generates random error situations to see that program using real squid system does survive those. Uses 2-3 COM ports. Usage SquidEmulator x z.. where x is 0 or 1 and indicates if Magnetometer and Demagnetizer are on same COM port. z... values are COM ports.
}
\classEvent{On New IO Message}{reads message and puts it in Buffer}
\closeClass

\beginField{messageBuffer}
\fieldDeclaration{private Stack messageBuffer}
\fieldComment{buffer for incoming messages}
\closeField

\beginField{online}
\fieldDeclaration{private bool online}
\fieldComment{indicates if system have been started}
\closeField

\beginField{acceleration}
\fieldDeclaration{private int acceleration}
\fieldComment{value between 0 and 127 default 5. Settings in the
20-50 range are usually employed.}
\closeField

\beginField{deceleration}
\fieldDeclaration{private int deceleration}
\fieldComment{value between 0 and 127 default 10. Settings in the
20-50 range are usually employed.}
\closeField

\beginField{velocity}
\fieldDeclaration{private int velocity}
\fieldComment{value between 50 and 12 000. The
decimal number issued is 10 times the actual pulse rate to the motor. Since the
motor requires 200 pulses (full step) or 400 pulses (half step) per revolution,
a speed setting of M10000 sets the motor to revolve at 5 revolutions per second
in full step or 2.5 revolutions in half step. This rate is one-half the sample
rate rotation due to the pulley ratios. The sample handler is set up at the
factory for half stepping.}
\closeField

\beginField{handlerStatus}
\fieldDeclaration{private String handlerStatus}
\fieldComment{5 end of move, previous G command complete, 7 hard limit stop, G motor is currently indexing}
\closeField

\beginField{commandedDistance}
\fieldDeclaration{private int commandedDistance}
\fieldComment{value between 1 and 16,777,215}
\closeField

\beginField{currentPosition}
\fieldDeclaration{private int currentPosition}
\fieldComment{value between 1 and 16,777,215}
\closeField

\beginField{homePosition}
\fieldDeclaration{private int homePosition}
\fieldComment{value between 1 and 16,777,215}
\closeField

\beginField{commandedRotation}
\fieldDeclaration{private int commandedRotation}
\fieldComment{angles are between 0 (0) and 2000 (360)}
\closeField

\beginField{currentRotation}
\fieldDeclaration{private int currentRotation}
\fieldComment{angles are between 0 (0) and 2000 (360)}
\closeField

\beginField{degausserCoil}
\fieldDeclaration{private int degausserCoil}
\fieldComment{(X, Y, Z) = (0,1,2) default axis Z}
\closeField

\beginField{degausserAmplitude}
\fieldDeclaration{private int degausserAmplitude}
\fieldComment{0->3000 default amp 0}
\closeField

\beginField{degausserDelay}
\fieldDeclaration{private int degausserDelay}
\fieldComment{1-9 seconds default delay 1 second}
\closeField

\beginField{degausserRamp}
\fieldDeclaration{private int degausserRamp}
\fieldComment{(3, 5, 7, 9) default 3}
\closeField

\beginField{degausserStatus}
\fieldDeclaration{private char degausserRamp}
\fieldComment{Z=Zero, T=Tracking, ?=Unknown}
\closeField

\beginField{messageReader}
\fieldDeclaration{private SerialIO[] messageReader}
\fieldComment{starts Threads which reads messages from selected COM port}
\closeField

\beginMethod{readMessageFromBuffer()}
\methodDeclaration{public void getSequences()}
\methodComment{
    reads message and commits it.
}
\closeMethod

\beginMethod{writeMessage(String message,int port)}
\methodDeclaration{public void writeMessage(String message,int port))}
\methodComment{
    send message to SerialIO to be sented.
}
\closeMethod

\beginMethod{run()}
\methodDeclaration{public void run())}
\methodComment{
    runs sequence where read data from buffer and run cheduled actions (move, rotate, demag, measure) and send feedback to COM ports.
}
\closeMethod
