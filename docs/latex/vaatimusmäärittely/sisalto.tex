\section{Introduction}

This document describes client requirements and system 
requirements for a SQUID magnetometer program that will be designed and 
implemeted as a software engineering student project at University of 
Helsinki at the Computer Science Department. The client is the Department 
of Geophysics.

This document serves as a contract between client and us..

Expected readership of this document here..

\subsection{Glossary}

Technical terms here..

\section{Overview}

A brief overview of the problem domain..

\section{User requirements definition}

Goals of the software set by client..

\subsection{Requirements}

Requirements by client..

\subsection{Restrictions}

Restrictions set by client..

\section{System requirements specification}

Specific explanation of the functions to be implemented

\subsection{Functional requirements}

\subsection{Non-functional requirements}

Requirements conserning the quality and performance of the 
software..

\subsubsection{Environment}

\subsubsection{Maintainability}

\subsubsection{Etc.}

\subsubsection{Etc.}

\subsection{External interfaces}

Interface to existing software and use of it described here..

\subsection{System restrictions}

\section{Use cases}

Describes planned use cases for the program. Derived from user interface prototype and requiremets. All use cases are made by "the user" in program main screen, unless otherwise noted.

\subsection{Measuring}

\subsubsection{Do single step measuring without demagnetization}

With any open project, insert as next step "0" or empty (default for new projects), meaning no demagnetization, and click "Single step".

Precondition: Sample in sample holder.\\
Postcondition: Sample measured, results on screen.\\
Error condition: The program shall let the user know if something went wrong.\\

\subsubsection{Do single step measuring with demagnetization}
\subsubsection{Do automatic demagnetization-measuring sequence}
  \begin{itemize}
  \item Pause automatic measuring sequence
  \item Cancel automatic measuring sequence
  \end{itemize}
\subsubsection{Do thellier measuring}
\subsubsection{Do thermal measuring}
\subsubsection{Measure magnetometer ground noise}
\subsubsection{Measure empty sample holder noise}
\subsubsection{Fully manual measuring}
  \begin{itemize}
  \item Move sample handler to desired position
  \item Rotate sample handler to desired angle
  \item Measure in current position
  \item Demagnetize in current position
  \end{itemize}

\subsection{File formats}

\subsubsection{Automatically save all measurement cycles in project (.dat?) file}
\subsubsection{Save standard sample measurement results in .std file}
\subsubsection{Export (thellier) results into .tdt file}
\subsubsection{Export single measurement details into .srm file}
\subsubsection{Print measurement results}
\subsubsection{Print graph sheet (with 7 different graphs; described elsewhere)}

\subsection{Functionality}

\subsubsection{Create new project (.dat file?)}
\subsubsection{Load project (.dat file?)}
\subsubsection{Append measurement results to project (.dat file?)}
\subsubsection{Panic abort operation instantly}

\subsection{AF sequences}

{\it As in automatic demagnetization-measuring sequences, or Alternating Field sequences}

\subsubsection{Insert AF sequence with start-step-stop values}
\subsubsection{Load AF sequence}
\subsubsection{Save AF sequence}
\subsubsection{Edit AF sequence on-the-fly}
\subsubsection{Edit stored AF sequences}
\subsubsection{Rename stored AF sequence}
\subsubsection{Delete stored AF sequence}

\section{User interface}

Overview of UI described here.. 

\section{Architecture overview}

\section{Validation}

Description of how to validate the set requirements.
