\section{Introduction}

This document describes client requirements and system 
requirements for a SQUID magnetometer program that will be designed and 
implemeted as a software engineering student project at University of 
Helsinki at the Computer Science Department. The client is the Department 
of Geophysics.

This document serves as a contract between client and us..

Expected readership of this document here..

\subsection{Glossary}

Technical terms here..

\section{Overview}

A brief overview of the problem domain..

\section{User requirements definition}

Goals of the software set by client..

\subsection{Requirements}

Requirements by client..

\subsection{Restrictions}

Restrictions set by client..

\section{System requirements specification}

Specific explanation of the functions to be implemented

\subsection{Functional requirements}

\subsection{Non-functional requirements}

Requirements conserning the quality and performance of the 
software..

\subsubsection{Environment}

\subsubsection{Maintainability}

\subsubsection{Etc.}

\subsubsection{Etc.}

\subsection{External interfaces}

Interface to existing software and use of it described here..

\subsection{System restrictions}

\section{Use cases}

% super-makrot pre-, post- ja error conditioneille x)
% huomaa ett� \error lopuksi ei ole uutta rivi�, joten laita se viimeiseksi
\newcommand{\pre}[1]{\textbf{Precondition:} #1\\}
\newcommand{\post}[1]{\textbf{Postcondition:} #1\\}
\newcommand{\error}[1]{\textbf{Error condition:} #1}

Describes planned use cases for the program. Derived from user interface prototype and requiremets. All use cases are made by "the user" in program main screen, unless otherwise noted.

\subsection{Measuring}

\subsubsection{Do single step measuring without demagnetization}

Enter as next AF demagnetization step "0" or empty (default for new projects), meaning no demagnetization, and click "Single step".

\pre{Open AF project, sample in sample holder.}
\post{Sample measured, results on screen.}
\error{The program shall let the user know if something went wrong.}

\subsubsection{Do single step measuring with demagnetization}

Enter as next AF demag step anything greater than zero, and click "Single step".

\pre{Open AF project, sample in sample holder.}
\post{Sample demagnetized (possibly ruined) and measured, results on screen.}
\error{The program shall let the user know if something went wrong, and, should the demagnetization field not be coming down, warn user with an alarm sound :)}

\subsubsection{Do automatic demagnetization-measuring sequence}

Enter the AF sequence (see \ref{sequsecaseaf} for ways to enter it) and click "Measure".

\pre{Open AF project, sample in sample holder.}
\post{Sample demagnetized according to entered AF sequence (possibly ruined) and measured after each demagnetization, results on screen.}
\error{The program shall let the user know if something went wrong, and, should the demagnetization field be uncalm, warn user with an alarm sound x)}

  \begin{itemize}

  \item Pause automatic measuring sequence

  While measure sequence is running, click "Pause".

  \pre{Ongoing measure sequence.}
  \post{Measure sequence halts after current step is done, results on screen.}
  \error{Program tells if sequence can't be paused (and something has gone terribly wrong).}

  \item Abort automatic measuring sequence

  While measure sequence is running or paused, click "Stop immediately".

  \pre{Ongoing or paused measure sequence.}
  \post{Measure sequence halts immediately [and program enters "fully manual" mode?]}
  \error{Program tells if sequence can't be aborted (and something has gone terribly wrong).}

  \end{itemize}

\subsubsection{Do thellier measuring}

Click "Single step". (Temperature can be entered later, as it won't affect measuring.)

\pre{Open TH project, sample in sample holder.}
\post{Sample measured, results on screen.}
\error{As usual.}

\subsubsection{Do thermal measuring}

[Exactly the same as thellier?]

Click "Single step". (Temperature can be entered later, as it won't affect measuring.)

\pre{Open TH project, sample in sample holder.}
\post{Sample measured, results on screen.}
\error{As usual.}

\subsubsection{Measure magnetometer ground noise}

Click "Noise" and "Calibrate".

\pre{None.}
\post{[Sample holder at home?] Ground noise measured, results on screen.}
\error{As usual.}

\subsubsection{Measure empty sample holder noise}

[2005-02-21 In current UI prototype, "Noise" actually does sample holder noise measuring.]

Click "Holder noise" and "Calibrate".

\pre{Empty sample holder.}
\post{[Sample holder at home?] Holder noise measured, results on screen.}
\error{As usual.}

\subsubsection{Fully manual measuring}

Click any of the manual control components [2005-02-21 at right third of UI proto].

\pre{None.}
\post{Manual action done, result on screen.}
\error{As usual.}

  \begin{itemize}
  \item Move sample handler to desired position
  \item Rotate sample handler to desired angle
  \item Measure in current position
  \item Demagnetize in current position
  \end{itemize}

\subsection{File formats}

\subsubsection{Automatically save all measurement cycles in project (.dat?) file}
\subsubsection{Save standard sample measurement results in .std file}
\subsubsection{Export (thellier) results into .tdt file}
\subsubsection{Export single measurement details into .srm file}
\subsubsection{Print measurement results}
\subsubsection{Print graph sheet (with 7 different graphs; described elsewhere)}

\subsection{Functionality}

\subsubsection{Create new project (.dat file?)}
\subsubsection{Load project (.dat file?)}
\subsubsection{Append measurement results to project (.dat file?)}
\subsubsection{Panic abort operation instantly}

When any measuring action, click "Stop immediately".

\pre{Single step measuring or ongoing measure sequence.}
\post{All demagnetization and measuring halts immediately [and program enters "fully manual" mode?]}
\error{Program tells if measuring can't be aborted, meaning something has gone terribly wrong.}

\subsection{AF sequences}
\label{sequsecaseaf}

{\it As in automatic demagnetization-measuring sequences, or Alternating Field sequences}

\subsubsection{Insert AF sequence with start-step-stop values}
\subsubsection{Load AF sequence}
\subsubsection{Save AF sequence}
\subsubsection{Edit AF sequence on-the-fly}
\subsubsection{Edit stored AF sequences}
\subsubsection{Rename stored AF sequence}
\subsubsection{Delete stored AF sequence}

\section{User interface}

Overview of UI described here.. 

\section{Architecture overview}

\section{Validation}

Description of how to validate the set requirements.
