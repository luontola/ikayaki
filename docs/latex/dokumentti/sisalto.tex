% Esimerkki oman komennon m��ritt�misest�. Jos komentoa tarvitaan useissa
% dokumenteissa, se kannattaa mielummin laittaa asetukset.tex-tiedostoon.
\newcommand{\tiedostonimi}[1]{\texttt{#1}}
\newcommand{\termi}[1]{\textit{#1}}

\section{Johdanto}

T�t� tiedostoa voi k�ytt�� pohjana Ohjelmistotuotantoprojekti-kurssilla
dokumentaatiota kirjoitettaessa. Tiedosto on tehty kev��n 2004 kurssia
varten.

\section{Esimerkkej�}

T�ss� luvussa n�ytet��n esimerkinomaisesti kuinka \LaTeX-ladontaj�rjestelm�n
ominaisuuksia k�ytet��n. Tiedostosta tiedostonimi{sisalto.tex} voi katsoa,
miten t�m� dokumentti on saatu aikaan. T�m� esimerkki ei ole tarkoitettu
kattavaksi johdannoksi LaTeX:iin, mutta onneksi verkosta l�ytyy kosolti
hyvi� ohjeita. Kirjastosta l�ytyv� Lamportin klassikko~\cite{lamport94} antaa
my�s hyv�n perustan LaTeX-ohjelman k�yt�lle.

\subsection{L�hdeviitteet}

Lue ohjelmistotuotantoalan
kirjallisuutta~\cite{pressman00}\cite{sommerville01},
tulet viisaammaksi.

\subsection{Taulukot}

Taulukossa~\ref{fig:aikataulukko} n�kyy, kuinka paljon aikaa eri projektien
tekemiseen on mennyt.

\begin{figure}[h]
\begin{center}
\begin{tabular}{ll}
Projekti & Kesto \\
\hline
A & 240 h \\
B & 300 h
\end{tabular}
\end{center}
\caption{Projekteihin k�ytetty aika}
\label{fig:aikataulukko}
\end{figure}

\subsection{Ranskalaiset viivat}

Karkeasti ottaen voidaan sanoa, ett� \termi{vaatimusm��rittely} koostuu
kahdesta vaiheesta, jotka ovat seuraavat.

\begin{enumerate}
\item Vaatimusten kartoitus
\item Vaatimusten analyysi
\end{enumerate}

\newpage

Ohjelmistotuotantoprojektin sidosryhmi� ovat

\begin{itemize}
\item opiskelijat,
\item ohjaajat,
\item asiakkaat ja
\item vastuuhenkil�t.
\end{itemize}

\subsection{Lainausmerkit}

Muistattehan, ett� ''sitaatit'' tuotetaan kahden per�kk�isen heittomerkin
avulla.
